\documentclass[12pt]{book}
\usepackage{geometry}
\usepackage[polutonikogreek, english]{babel}
\usepackage{microtype}
\usepackage{titlesec}
\usepackage{fancyhdr}
\usepackage{lettrine}
\usepackage{kpfonts}
\usepackage{tocloft}

% page geometry

\geometry{
    paperwidth=5.562in,
    paperheight=9in,
    inner=0.75in,
    outer=0.75in,
    top=1in,
    bottom=1in,
    headsep=0.25in,
    footskip=0.35in
}

% header and footer style
\pagestyle{fancy}
\fancyhf{}
\fancyhead[LE, RO]{\thepage}
\fancyhead[RE]{\nouppercase{\leftmark}}
\fancyhead[LO]{\nouppercase{\rightmark}}

% section style

\titleformat{\section}[block]
    {\normalfont\bfseries\Large}
    {\thesection}
    {1em}
    {}

\titlespacing*{\section}{0pt}{20pt}{20pt}

\renewcommand{\LettrineFontHook}{\kpfamily}

\newcommand{\partnonum}[2]{
    \part*{#2}
    \addcontentsline{toc}{part}{#1 \hspace{1em} #2}
    \markboth{#1 #2}{}
}

\newcommand{\sectionnonum}[3][\empty]{
    \section*{#3}
    \addcontentsline{toc}{section}{#1 \hspace{1em} #3}
    \ifx\empty#1\relax
        \markright{#3}
    \else
        \markright{#1}
    \fi
}


\title{Florilegium Patria // Volume I}

\begin{document}

\frontmatter

\begin{titlepage}
    \begin{center}
        \vspace*{2in}
        {\Huge\bfseries Florilegium Patria}

        \vspace{0.5in}
        {\Large Dominic Cook}

        \vspace{2in}
        {}
        {\Large}
        {\Large MMXXIV}
    \end{center}
\end{titlepage}

\tableofcontents

\mainmatter

%mainmatter: parts and chapters

\partnonum{1}{CLEMENT}

\sectionnonum{1}{Introductory Note to the First Epistle of Clement to the Corinthians [A.D. 30–100.]}
CLEMENT was probably a Gentile and a Roman. He seems to have been at Philippi
with St. Paul (A.D. 57) when that first-born of the Western churches was passing through great trials
of faith. There, with holy women and others, he ministered to the apostle and to the saints. As this
city was a Roman colony, we need not inquire how a Roman happened to be there. He was possibly
in some public service, and it is not improbable that he had visited Corinth in those days. From the
apostle, and his companion, St. Luke, he had no doubt learned the use of the Septuagint, in which
his knowledge of the Greek tongue soon rendered him an adept. His copy of that version, however,
does not always agree with the Received Text, as the reader will perceive.
A co-presbyter with Linus and Cletus, he succeeded them in the government of the Roman
Church. I have reluctantly adopted the opinion that his Epistle was written near the close of his
life, and not just after the persecution of Nero. It is not improbable that Linus and Cletus both
perished in that fiery trial, and that Clement’s immediate succession to their work and place occasions
the chronological difficulties of the period. After the death of the apostles, for the Roman
imprisonment and martyrdom of St. Peter seem historical, Clement was the natural representative
of St. Paul, and even of his companion, the “apostle of the circumcision;” and naturally he wrote
the Epistle in the name of the local church, when brethren looked to them for advice. St. John, no
doubt, was still surviving at Patmos or in Ephesus; but the Philippians, whose intercourse with
Rome is attested by the visit of Epaphroditus, looked naturally to the surviving friends of their great
founder; nor was the aged apostle in the East equally accessible. All roads pointed towards the
Imperial City, and started from its Milliarium Aureum. But, though Clement doubtless wrote the
letter, he conceals his own name, and puts forth the brethren, who seem to have met in council, and
sent a brotherly delegation (Chap. lix.). The entire absence of the spirit of Diotrephes (3 John 9),
and the close accordance of the Epistle, in humility and meekness, with that of St. Peter (1 Pet. v.
1–5), are noteworthy features. The whole will be found animated with the loving and faithful spirit
of St. Paul’s dear Philippians, among whom the writer had learned the Gospel.
Clement fell asleep, probably soon after he despatched his letter. It is the legacy of one who
reflects the apostolic age in all the beauty and evangelical truth which were the first-fruits of the
Spirit’s presence with the Church. He shares with others the aureole of glory attributed by St. Paul
(Phil. iv. 3), “His name is in the Book of Life.”
The plan of this publication does not permit the restoration, in this volume, of the recently
discovered portions of his work. It is the purpose of the editor to present this, however, with other
recently discovered relics of primitive antiquity, in a supplementary volume, should the undertaking
meet with sufficient encouragement. The so-called second Epistle of Clement is now known to be
the work of another, and has been relegated to another place in this series.


\sectionnonum{2}{The following is the INTRODUCTORY NOTICE of the original editors and translators, Drs. Roberts
and Donaldson}
THE first Epistle, bearing the name of Clement, has been preserved to us in a single manuscript
only. Though very frequently referred to by ancient Christian writers, it remained unknown to the
scholars of Western Europe until happily discovered in the Alexandrian manuscript. This MS. of
the Sacred Scriptures (known and generally referred to as Codex A) was presented in 1628 by Cyril,
Patriarch of Constantinople, to Charles I., and is now preserved in the British Museum. Subjoined
to the books of the New Testament contained in it, there are two writings described as the Epistles
of one Clement. Of these, that now before us is the first. It is tolerably perfect, but there are many
slight lacunæ, or gaps, in the MS., and one whole leaf is supposed to have been lost towards the
close. These lacunæ, however, so numerous in some chapters, do not generally extend beyond a
word or syllable, and can for the most part be easily supplied.
Who the Clement was to whom these writings are ascribed, cannot with absolute certainty be
determined. The general opinion is, that he is the same as the person of that name referred to by
St. Paul (Phil. iv. 3). The writings themselves contain no statement as to their author. The first, and
by far the longer of them, simply purports to have been written in the name of the Church at Rome
to the Church at Corinth. But in the catalogue of contents prefixed to the MS. they are both plainly
attributed to one Clement; and the judgment of most scholars is, that, in regard to the first Epistle
at least, this statement is correct, and that it is to be regarded as an authentic production of the friend
and fellow-worker of St. Paul. This belief may be traced to an early period in the history of the
Church. It is found in the writings of Eusebius (Hist. Eccl., iii. 15), of Origen (Comm. in Joan., i.
29), and others. The internal evidence also tends to support this opinion. The doctrine, style, and
manner of thought are all in accordance with it; so that, although, as has been said, positive certainty
cannot be reached on the subject, we may with great probability conclude that we have in this
Epistle a composition of that Clement who is known to us from Scripture as having been an associate
of the great apostle.
The date of this Epistle has been the subject of considerable controversy. It is clear from the
writing itself that it was composed soon after some persecution (chap. i.) which the Roman Church
had endured; and the only question is, whether we are to fix upon the persecution under Nero or
Domitian. If the former, the date will be about the year 68; if the latter, we must place it towards
the close of the first century or the beginning of the second. We possess no external aid to the
settlement of this question. The lists of early Roman bishops are in hopeless confusion, some making
Clement the immediate successor of St. Peter, others placing Linus, and others still Linus and
Anacletus, between him and the apostle. The internal evidence, again, leaves the matter doubtful,
though it has been strongly pressed on both sides. The probability seems, on the whole, to be in
favour of the Domitian period, so that the Epistle may be dated about A.D. 97.

\end{document}











ANF01. The Apostolic Fathers with Justin Martyr and Irenaeus Philip Schaff
This Epistle was held in very great esteem by the early Church. The account given of it by
Eusebius (Hist. Eccl., iii. 16) is as follows: “There is one acknowledged Epistle of this Clement
(whom he has just identified with the friend of St. Paul), great and admirable, which he wrote in
the name of the Church of Rome to the Church at Corinth, sedition having then arisen in the latter
Church. We are aware that this Epistle has been publicly read in very many churches both in old
times, and also in our own day.” The Epistle before us thus appears to have been read in numerous
churches, as being almost on a level with the canonical writings. And its place in the Alexandrian
MS., immediately after the inspired books, is in harmony with the position thus assigned it in the
3
primitive Church. There does indeed appear a great difference between it and the inspired writings
in many respects, such as the fanciful use sometimes made of Old-Testament statements, the fabulous
stories which are accepted by its author, and the general diffuseness and feebleness of style by
which it is distinguished. But the high tone of evangelical truth which pervades it, the simple and
earnest appeals which it makes to the heart and conscience, and the anxiety which its writer so
constantly shows to promote the best interests of the Church of Christ, still impart an undying charm
to this precious relic of later apostolic times.
[N.B.—A sufficient guide to the recent literature of the Clementine MSS. and discoveries may
be found in The Princeton Review, 1877, p. 325, also in Bishop Wordsworth’s succinct but learned
Church History to the Council of Nicæa, p. 84. The invaluable edition of the Patres Apostolici, by
Jacobson (Oxford, 1840), with a critical text and rich prolegomena and annotations, cannot be
dispensed with by any Patristic inquirer. A. C. C.]
4


\partnonum{1}{The First Epistle of Clement to the Corinthians}

\sectionnonum[Chapter I]{1}{The salutation. Praise of the Corinthians before the breaking forth of schism among them. \protect\footnote{In the only known MS. of this Epistle, the title is thus given at the close.}}
THE Church of God which sojourns at Rome, to the Church of God sojourning at Corinth, to
them that are called and sanctified by the will of God, through our Lord Jesus Christ: Grace unto
you, and peace, from Almighty God through Jesus Christ, be multiplied.
Owing, dear brethren, to the sudden and successive calamitous events which have happened to
ourselves, we feel that we have been somewhat tardy in turning our attention to the points respecting
which you consulted us;\footnote{[Note the fact that the Corinthians asked this of their brethren, the personal friends of their apostle St. Paul. Clement’s own name does not appear in this Epistle.]} and especially to that shameful and detestable sedition, utterly abhorrent
to the elect of God, which a few rash and self-confident persons have kindled to such a pitch of
frenzy, that your venerable and illustrious name, worthy to be universally loved, has suffered
grievous injury.\footnote{Literally, ``is greatly blasphemed.''} For who ever dwelt even for a short time among you, and did not find your faith
to be as fruitful of virtue as it was firmly established?\footnote{Literally, ``did not prove your all-virtuous and firm faith.''}
Who did not admire the sobriety and
moderation of your godliness in Christ? Who did not proclaim the magnificence of your habitual
hospitality? And who did not rejoice over your perfect and well-grounded knowledge? For ye did
all things without respect of persons, and walked in the commandments of God, being obedient to
those who had the rule over you, and giving all fitting honour to the presbyters among you. Ye
enjoined young men to be of a sober and serious mind; ye instructed your wives to do all things
with a blameless, becoming, and pure conscience, loving their husbands as in duty bound; and ye
taught them that, living in the rule of obedience, they should manage their household affairs
becomingly, and be in every respect marked by discretion.

\sectionnonum[Chapter II]{2}{Praise of the Corinthians continued.}
Moreover, ye were all distinguished by humility, and were in no respect puffed up with pride,
but yielded obedience rather than extorted it,\footnote{Eph. v. 21; 1 Pet. v. 5.} and were more willing to give than to receive.\footnote{Acts xx. 35.} 
Content with the provision which God had made for you, and carefully attending to His words, ye were
inwardly filled\footnote{Literally, ``ye embraced it in your bowels.'' [Concerning the complaints of Photius (ninth century) against Clement, see
Bull's Defensio Fidei Nic{\ae}n{\ae}, Works, vol. v. p. 132.]} with His doctrine, and His sufferings were before your eyes. Thus a profound and
abundant peace was given to you all, and ye had an insatiable desire for doing good, while a full
outpouring of the Holy Spirit was upon you all. Full of holy designs, ye did, with true earnestness
of mind and a godly confidence, stretch forth your hands to God Almighty, beseeching Him to be
merciful unto you, if ye had been guilty of any involuntary transgression. Day and night ye were
anxious for the whole brotherhood,\footnote{1 Pet. ii. 17.} that the number of God’s elect might be saved with mercy and
a good conscience.\footnote{So, in the MS., but many have suspected that the text is here corrupt. Perhaps the best emendation is that which substitutes \textgreek{sunais}\texttheta\textgreek{\'{h}sewz}, ``compassion,'' for \textgreek{suneid}\texttheta\textgreek{\'{h}sewz}, ``conscience.''} 
Ye were sincere and uncorrupted, and forgetful of injuries between one another.
Every kind of faction and schism was abominable in your sight. Ye mourned over the transgressions
of your neighbours: their deficiencies you deemed your own. Ye never grudged any act of kindness,
being ``ready to every good work.''\footnote{Tit. iii. 1.}Adorned by a thoroughly virtuous and religious life, ye did
all things in the fear of God. The commandments and ordinances of the Lord were written upon
the tablets of your hearts.\footnote{Prov. vii. 3.}

\sectionnonum[Chapter III]{3}{The sad state of the Corinthian church after sedition arose in it from envy and
emulation.}
Every kind of honour and happiness\footnote{Literally, ``enlargement''} was bestowed upon you, and then was fulfilled that which
is written, ``My beloved did eat and drink, and was enlarged and became fat, and kicked.''\footnote{Deut. xxxii. 15.}
Hence
flowed emulation and envy, strife and sedition, persecution and disorder, war and captivity. So the
worthless rose up against the honoured, those of no reputation against such as were renowned, the
foolish against the wise, the young against those advanced in years. For this rea son righteousness
and peace are now far departed from you, inasmuch as every one abandons the fear of God, and is
become blind in His faith, \footnote{It seems necessary to refer \textgreek{anto\~{n}} to God, in opposition to the translation given by Abp. Wake and others.}
neither walks in the ordinances of His appointment, nor acts a part
becoming a Christian,\footnote{Literally, ``Christ;'' comp. 2 Cor. i. 21, Eph. iv. 20.}
but walks after his own wicked lusts, resuming the practice of an unrighteous
and ungodly envy, by which death itself entered into the world.\footnote{Wisdom ii. 24.}

\sectionnonum[Chapter IV]{4}{Many evils have already flowed from this source in ancient times.}
For thus it is written: ``And it came to pass after certain days, that Cain brought of the fruits of
the earth a sacrifice unto God; and Abel also brought of the firstlings of his sheep, and of the fat
thereof. And God had respect to Abel and to his offerings, but Cain and his sacrifices He did not
regard. And Cain was deeply grieved, and his countenance fell. And God said to Cain, Why art
thou grieved, and why is thy countenance fallen? If thou offerest rightly, but dost not divide rightly,
hast thou not sinned? Be at peace: thine offering returns to thyself, and thou shalt again possess it.
And Cain said to Abel his brother, Let us go into the field. And it came to pass, while they were in
the field, that Cain rose up against Abel his brother, and slew him.''\footnote{Gen. iv. 3 - 8. The writer here, as always, follows the reading of the Septuagint, which in this passage both alters and adds
to the Hebrew text. We have given the rendering approved by the best critics; but some prefer to translate, as in our English
version, ``unto thee shall be his desire, and thou shalt rule over him.'' See, for an ancient explanation of the passage, Iren{\ae}us,
Adv. H{\ae}r., iv. 18, 3.} Ye see, brethren, how envy
and jealousy led to the murder of a brother. Through envy, also, our father Jacob fled from the face
of Esau his brother.\footnote{Gen. xxvii. 41, etc.} Envy made Joseph be persecuted unto death, and to come into bondage.\footnote{Gen. xxxvii.}
Envy compelled Moses to flee from the face of Pharaoh king of Egypt, when he heard these words
from his fellow-countryman, ``Who made thee a judge or a ruler over us? wilt thou kill me, as thou
didst kill the Egyptian yesterday?''\footnote{Ex. ii. 14.} On account of envy, Aaron and Miriam had to make their
abode without the camp.\footnote{Num. xii. 14, 15. [In our copies of the Septuagint this is not affirmed of Aaron.]}
Envy brought down Dathan and Abiram alive to Hades, through the
sedition which they excited against God’s servant Moses.\footnote{Num. xvi. 33.}
Through envy, David underwent the
hatred not only of foreigners, but was also persecuted by Saul king of Israel.\footnote{1 Kings xviii. 8, etc.}

\sectionnonum[Chapter V]{5}{No less evils have arisen from the same source in the most recent times. The
martyrdom of Peter and Paul.}
But not to dwell upon ancient examples, let us come to the most recent spiritual heroes.\footnote{Literally, ``those who have been athletes.''}
Let us take the noble examples furnished in our own generation. Through envy and jealousy, the greatest
and most righteous pillars [of the Church] have been persecuted and put to death.\footnote{Some fill up the lacuna here found in the MS. so as to read, ``have come to a grievous death.''}
Let us set before our eyes the illustrious\footnote{Literally, ``good.'' [The martyrdom of St. Peter is all that is thus connected with his arrival in Rome. His numerous labours were restricted to the Circumcision.]}
apostles. Peter, through unrighteous envy, endured not one or two, but
numerous labours and when he had at length suffered martyrdom, departed to the place of glory
due to him. Owing to envy, Paul also obtained the reward of patient endurance, after being seven
times thrown into captivity,\footnote{Seven imprisonments of St. Paul are not referred to in Scripture.}
compelled\footnote{Archbishop Wake here reads ``scourged.'' We have followed the most recent critics in filling up the numerous lacunæ in this chapter.}
to flee, and stoned. After preaching both in the east and
west, he gained the illustrious reputation due to his faith, having taught righteousness to the whole
world, and come to the extreme limit of the west,\footnote{Some think Rome, others Spain, and others even Britain, to be here referred to. [See note at end.]}
and suffered martyrdom under the prefects.\footnote{That is, under Tigellinus and Sabinus, in the last year of the Emperor Nero; but some think Helius and Polycletus are referred to; and others, both here and in the preceding sentence, regard the words as denoting simply the witness borne by Peter and Paul to the truth of the gospel before the rulers of the earth.}
Thus was he removed from the world, and went into the holy place, having proved himself a striking
example of patience.

\sectionnonum[Chapter VI]{6}{Continuation. Several other martyrs.}
To these men who spent their lives in the practice of holiness, there is to be added a great
multitude of the elect, who, having through envy endured many indignities and tortures, furnished
us with a most excellent example. Through envy, those women, the Danaids\footnote{Some suppose these to have been the names of two eminent female martyrs under Nero; others regard the clause as an interpolation. [Many ingenious conjectures might be cited; but see Jacobson’s valuable note, Patres Apostol., vol. i. p. 30.]}and Dirc{\ae}, being
persecuted, after they had suffered terrible and unspeakable torments, finished the course of their
faith with stedfastness,\footnote{Literally, ``have reached to the stedfast course of faith.''}and though weak in body, received a noble reward. Envy has alienated
wives from their husbands, and changed that saying of our father Adam, ``This is now bone of my
bones, and flesh of my flesh.''\footnote{Gen. ii. 23.}Envy and strife have overthrown great cities and rooted up mighty
nations.

\sectionnonum[Chapter VII]{7}{An exhortation to repentance.}
These things, beloved, we write unto you, not merely to admonish you of your duty, but also
to remind ourselves. For we are struggling on the same arena, and the same conflict is assigned to
both of us. Wherefore let us give up vain and fruitless cares, and approach to the glorious and
venerable rule of our holy calling. Let us attend to what is good, pleasing, and acceptable in the
sight of Him who formed us. Let us look stedfastly to the blood of Christ, and see how precious
that blood is to God,\footnote{Some insert ``Father.''}which, having been shed for our salvation, has set the grace of repentance
before the whole world. Let us turn to every age that has passed, and learn that, from generation
to generation, the Lord has granted a place of repentance to all such as would be converted unto
Him. Noah preached repentance, and as many as listened to him were saved.\footnote{Gen. vii.; 1 Pet. iii. 20; 2 Pet. ii. 5.}Jonah proclaimed
destruction to the Ninevites;\footnote{Jon. iii.}but they, repenting of their sins, propitiated God by prayer, and
obtained salvation, although they were aliens [to the covenant] of God.

\sectionnonum[Chapter VIII]{8}{Continuation respecting repentance.}
The ministers of the grace of God have, by the Holy Spirit, spoken of repentance; and the Lord
of all things has himself declared with an oath regarding it, ``As I live, saith the Lord, I desire not
the death of the sinner, but rather his repentance;''\footnote{Ezek. xxxiii. 11.}adding, moreover, this gracious declaration,
``Repent, O house of Israel, of your iniquity.\footnote{Ezek. xviii. 30.}Say to the children of My people, Though your sins
reach from earth to heaven, I and though they be redder\footnote{Comp. Isa. i. 18.}than scarlet, and blacker than sackcloth,
yet if ye turn to Me with your whole heart, and say, Father! I will listen to you, as to a holy\footnote{These words are not found in Scripture, though they are quoted again by Clem. Alex. (P{\ae}dag., i. 10) as from Ezekiel.}people.``
And in another place He speaks thus: ``Wash you, and become clean; put away the wickedness of
your souls from before mine eyes; cease from your evil ways, and learn to do well; seek out
judgment, deliver the oppressed, judge the fatherless, and see that justice is done to the widow; and
come, and let us reason together. He declares, Though your sins be like crimson, I will make them
white as snow; though they be like scarlet, I will whiten them like wool. And if ye be willing and
obey Me, ye shall eat the good of the land; but if ye refuse, and will not hearken unto Me, the sword
shall devour you, for the mouth of the Lord hath spoken these things.''\footnote{Isa. i. 16 - 20.}Desiring, therefore, that
all His beloved should be partakers of repentance, He has, by His almighty will, established [these
declarations].

\sectionnonum[Chapter IX]{9}{Examples of the saints.}
Wherefore, let us yield obedience to His excellent and glorious will; and imploring His mercy
and loving-kindness, while we forsake all fruitless labours,\footnote{Some read \textgreek{mataiolog\'{i}an}, ``vain talk.''}and strife, and envy, which leads to
death, let us turn and have recourse to His compassions. Let us stedfastly contemplate those who
have perfectly ministered to His excellent glory. Let us take (for instance) Enoch, who, being found
righteous in obedience, was translated, and death was never known to happen to him.\footnote{Gen. v. 24; Heb. xi. 5. Literally, ``and his death was not found.''}Noah, being
found faithful, preached regeneration to the world through his ministry; and the Lord saved by him
the animals which, with one accord, entered into the ark.

\sectionnonum[Chapter X]{10}{Continuation of the above.}
Abraham, styled ``the friend,''\footnote{Isa. xli. 8; 2 Chron. xx. 7; Judith viii. 19; Jas. ii. 23.}was found faithful, inasmuch as he rendered obedience to the
words of God. He, in the exercise of obedience, went out from his own country, and from his
kindred, and from his father’s house, in order that, by forsaking a small territory, and a weak family,
and an insignificant house, he might inherit the promises of God. For God said to him, ``Get thee
out from thy country, and from thy kindred, and from thy father’s house, into the land which I shall
show thee. And I will make thee a great nation, and will bless thee, and make thy name great, and
thou shall be blessed. And I will bless them that bless thee, and curse them that curse thee; and in
thee shall all the families of the earth be blessed.''\footnote{Gen. xii. 1 - 3.}And again, on his departing from Lot, God said
to him. ``Lift up thine eyes, and look from the place where thou now art, northward, and southward,
and eastward, and westward; for all the land which thou seest, to thee will I give it, and to thy seed
for ever. And I will make thy seed as the dust of the earth, [so that] if a man can number the dust
of the earth, then shall thy seed also be numbered.''\footnote{Gen. xiii. 14 - 16.}And again [the Scripture] saith, ``God brought
forth Abram, and spake unto him, Look up now to heaven, and count the stars if thou be able to
number them; so shall thy seed be. And Abram believed God, and it was counted to him for
righteousness.''\footnote{Gen. xv. 5, 6; Rom. iv. 3.}On account of his faith and hospitality, a son was given him in his old age; and
in the exercise of obedience, he offered him as a sacrifice to God on one of the mountains which
He showed him.\footnote{Gen. xxi. 22; Heb. xi. 17.}

\sectionnonum[Chapter XI]{11}{Continuation. Lot.}
On account of his hospitality and godliness, Lot was saved out of Sodom when all the country
round was punished by means of fire and brimstone, the Lord thus making it manifest that He does
not forsake those that hope in Him, but gives up such as depart from Him to punishment and torture.\footnote{Gen. xix.; comp. 2 Pet. ii. 6 - 9.}
For Lot’s wife, who went forth with him, being of a different mind from himself and not continuing
in agreement with him [as to the command which had been given them], was made an example of,
so as to be a pillar of salt unto this day.\footnote{So Joseph., Antiq., i. 11, 4; Iren{\ae}us, Adv. H{\ae}r., iv. 31.}This was done that all might know that those who are of
a double mind, and who distrust the power of God, bring down judgment on themselves\footnote{Literally, ``become a judgment and sign.``}and
become a sign to all succeeding generations.

\sectionnonum[Chapter XII]{12}{The rewards of faith and hospitality. Rahab.}
On account of her faith and hospitality, Rahab the harlot was saved. For when spies were sent
by Joshua, the son of Nun, to Jericho, the king of the country ascertained that they were come to
spy out their land, and sent men to seize them, in order that, when taken, they might be put to death.
But the hospitable Rahab receiving them, concealed them on the roof of her house under some
stalks of flax. And when the men sent by the king arrived and said ``There came men unto thee who
are to spy out our land; bring them forth, for so the king commands,'' she answered them, ``The two
men whom ye seek came unto me, but quickly departed again and are gone,'' thus not discovering
the spies to them. Then she said to the men, ``I know assuredly that the Lord your God hath given
you this city, for the fear and dread of you have fallen on its inhabitants. When therefore ye shall
have taken it, keep ye me and the house of my father in safety.`` And they said to her, ``It shall be
as thou hast spoken to us. As soon, therefore, as thou knowest that we are at hand, thou shall gather
all thy family under thy roof, and they shall be preserved, but all that are found outside of thy
dwelling shall perish.''\footnote{Josh. ii.; Heb. xi. 31.}Moreover, they gave her a sign to this effect, that she should hang forth
from her house a scarlet thread. And thus they made it manifest that redemption should flow through
the blood of the Lord to all them that believe and hope in God.\footnote{Others of the Fathers adopt the same allegorical interpretation, e.g., Justin Mar., Dial. c. Tryph., n. 111; Iren{\ae}us, Adv.H{\ae}r., iv. 20. [The whole matter of symbolism under the law must be more thoroughly studied if we would account for such strong language as is here applied to a poetical or rhetorical figure.]}Ye see, beloved, that there was
not only faith, but prophecy, in this woman.

\sectionnonum[Chapter XIII]{13}{An exhortation to humility.}
Let us therefore, brethren, be of humble mind, laying aside all haughtiness, and pride, and
foolishness, and angry feelings; and let us act according to that which is written (for the Holy Spirit
saith, ``Let not the wise man glory in his wisdom, neither let the mighty man glory in his might,
neither let the rich man glory in his riches; but let him that glorieth glory in the Lord, in diligently
seeking Him, and doing judgment and righteousness''\footnote{Jer. ix.  23, 24; 1 Cor. i.  31; 2 Cor. x.  17.}), being especially mindful of the words of
the Lord Jesus which He spake, teaching us meekness and long-suffering. For thus He spoke: ``Be
ye merciful, that ye may obtain mercy; forgive, that it may be forgiven to you; as ye do, so shall it
be done unto you; as ye judge, so shall ye be judged; as ye are kind, so shall kindness be shown to
you; with what measure ye mete, with the same it shall be measured to you.''\footnote{Comp. Matt. vi.  12 -  15, Matt. vii.  2; Luke vi.  36 -  38. }By this precept and
by these rules let us establish ourselves, that we walk with all humility in obedience to His holy
words. For the holy word saith, ``On whom shall I look, but on him that is meek and peaceable, and
that trembleth at My words?''\footnote{Isa. lxvi.  2.}

\sectionnonum[Chapter XIV]{14}{We should obey God rather than the authors of sedition.}
It is right and holy therefore, men and brethren, rather to obey God than to follow those who,
through pride and sedition, have become the leaders of a detestable emulation. For we shall incur
no slight injury, but rather great danger, if we rashly yield ourselves to the inclinations of men who
aim at exciting strife and tumults, so as to draw us away from what is good. Let us be kind one to
another after the pattern of the tender mercy and benignity of our Creator. For it is written, ``The
kind-hearted shall inhabit the land, and the guiltless shall be left upon it, but transgressors shall be
destroyed from off the face of it.''\footnote{Prov. ii.  21, 22.  }And again [the Scripture] saith, ``I saw the ungodly highly
exalted, and lifted up like the cedars of Lebanon: I passed by, and, behold, he was not; and I diligently
sought his place, and could not find it. Preserve innocence, and look on equity: for there shall be
a remnant to the peaceful man.''\footnote{Ps. xxxvii.  35 -  37. ``Remnant'' probably refers either to the memory or posterity of the righteous.  }

\sectionnonum[Chapter XV]{15}{We must adhere to those who cultivate peace, not to those who merely pretend
to do so.}
Let us cleave, therefore, to those who cultivate peace with godliness, and not to those who
hypocritically profess to desire it. For [the Scripture] saith in a certain place, ``This people honoureth
Me with their lips, but their heart is far from Me.''\footnote{Isa. xxix.  13; Matt. xv.  8; Mark vii.  6.  }And again: ``They bless with their mouth, but
curse with their heart.''\footnote{Ps. lxii.  4.  }And again it saith, ``They loved Him with their mouth, and lied to Him
with their tongue; but their heart was not right with Him, neither were they faithful in His covenant.''\footnote{Ps. lxxviii.  36, 37.  }``Let the deceitful lips become silent,''\footnote{Ps. xxxi.  18.  }[and ``let the Lord destroy all the lying lips,\footnote{These words within brackets are not found in the MS., but have been inserted from the Septuagint by most editors.  }] and the
boastful tongue of those who have said, Let us magnify our tongue; our lips are our own; who is
lord over us? For the oppression of the poor, and for the sighing of the needy, will I now arise, saith
the Lord: I will place him in safety; I will deal confidently with him.''\footnote{Ps. xii.  3 -  5.  }

\sectionnonum[Chapter XVI]{16}{Christ as an example of humility.}
For Christ is of those who are humble-minded, and not of those who exalt themselves over His
flock. Our Lord Jesus Christ, the Sceptre of the majesty of God, did not come in the pomp of pride
or arrogance, although He might have done so, but in a lowly condition, as the Holy Spirit had
declared regarding Him. For He says, ``Lord, who hath believed our report, and to whom is the arm
of the Lord revealed? We have declared [our message] in His presence: He is, as it were, a child,
and like a root in thirsty ground; He has no form nor glory, yea, we saw Him, and He had no form
nor comeliness; but His form was without eminence, yea, deficient in comparison with the [ordinary]
form of men. He is a man exposed to stripes and suffering, and acquainted with the endurance of
grief: for His countenance was turned away; He was despised, and not esteemed. He bears our
iniquities, and is in sorrow for our sakes; yet we supposed that [on His own account] He was exposed
to labour, and stripes, and affliction. But He was wounded for our transgressions, and bruised for
our iniquities. The chastisement of our peace was upon Him, and by His stripes we were healed.
All we, like sheep, have gone astray; [every] man has wandered in his own way; and the Lord has
delivered Him up for our sins, while He in the midst of His sufferings openeth not His mouth. He
was brought as a sheep to the slaughter, and as a lamb before her shearer is dumb, so He openeth
not His mouth. In His humiliation His judgment was taken away; who shall declare His generation?
for His life is taken from the earth. For the transgressions of my people was He brought down to
death. And I will give the wicked for His sepulchre, and the rich for His death,\footnote{The Latin of Cotelerius, adopted by Hefele and Dressel, translates this clause as follows: ``I will set free the wicked on account of His sepulchre, and the rich on account of His death.'' }because He did
no iniquity, neither was guile found in His mouth. And the Lord is pleased to purify Him by stripes.\footnote{The reading of the MS. is \textgreek{t\~{h}z plhg\~{h}z}, ``purify, or free, Him from stripes.'' We have adopted the emendation of Junius.  }If ye make\footnote{Wotton reads, ``If He make.'' }an offering for sin, your soul shall see a long-lived seed. And the Lord is pleased to
relieve Him of the affliction of His soul, to show Him light, and to form Him with understanding,\footnote{Or, ``fill Him with understanding,'' if \textgreek{pl\~{h}sai} should be read instead of \textgreek{pl\'{a}sai}, as Grabe suggests.  }to justify the Just One who ministereth well to many; and He Himself shall carry their sins. On this
account He shall inherit many, and shall divide the spoil of the strong; because His soul was delivered
to death, and He was reckoned among the transgressors, and He bare the sins of many, and for their
sins was He delivered.''\footnote{Isa. liii. The reader will observe how often the text of the Septuagint, here quoted, differs from the Hebrew as represented by our authorized English version.  }And again He saith, ``I am a worm, and no man; a reproach of men, and
despised of the people. All that see Me have derided Me; they have spoken with their lips; they
have wagged their head, [saying] He hoped in God, let Him deliver Him, let Him save Him, since
He delighteth in Him.''\footnote{Ps. xxii.  6 -  8.  }Ye see, beloved, what is the example which has been given us; for if the
Lord thus humbled Himself, what shall we do who have through Him come under the yoke of His
grace?

\sectionnonum[Chapter XVII]{17}{The saints as examples of humility.}
Let us be imitators also of those who in goat-skins and sheep-skins\footnote{Heb. xi.  37.  }went about proclaiming
the coming of Christ; I mean Elijah, Elisha, and Ezekiel among the prophets, with those others to
whom a like testimony is borne [in Scripture]. Abraham was specially honoured, and was called
the friend of God; yet he, earnestly regarding the glory of God, humbly declared, ``I am but dust
and ashes.''\footnote{Gen. xviii.  27.}Moreover, it is thus written of Job, ``Job was a righteous man, and blameless, truthful,
God-fearing, and one that kept himself from all evil.''\footnote{Job i.  1.  }But bringing an accusation against himself, 
he said, ``No man is free from defilement, even if his life be but of one day.''\footnote{Job xiv.  4, 5. [Septuagint.] }Moses was called 
faithful in all God’s house;\footnote{Num. xii.  7; Heb. iii.  2.  }and through his instrumentality, God punished Egypt \footnote{Some fill up the lacuna which here occurs in the MS. by ``Israel.''}with plagues 
and tortures. Yet he, though thus greatly honoured, did not adopt lofty language, but said, when
the divine oracle came to him out of the bush, ``Who am I, that Thou sendest me? I am a man of a
feeble voice and a slow tongue.''\footnote{Ex. iii.  11, Ex. iv.  10.  }And again he said, ``I am but as the smoke of a pot.'' \footnote{This is not found in Scripture. [They were probably in Clement’s version. Comp. Ps. cxix.  83.] }

\sectionnonum[Chapter XVIII]{18}{David as an example of humility.}
But what shall we say concerning David, to whom such testimony was borne, and of whom\footnote{Or, as some render, ``to whom.''}
God said, ``I have found a man after Mine own heart, David the son of Jesse; and in everlasting
mercy have I anointed him?''\footnote{Ps. lxxxix.  21.}Yet this very man saith to God, ``Have mercy on me, O Lord,
according to Thy great mercy; and according to the multitude of Thy compassions, blot out my
transgression. Wash me still more from mine iniquity, and cleanse me from my sin. For I
acknowledge my iniquity, and my sin is ever before me. Against Thee only have I sinned, and done
that which was evil in Thy sight; that Thou mayest be justified in Thy sayings, and mayest overcome
when Thou\footnote{Or, ``when Thou judgest.`` }art judged. For, behold, I was conceived in transgressions, and in my sins did my
mother conceive me. For, behold, Thou hast loved truth; the secret and hidden things of wisdom
hast Thou shown me. Thou shalt sprinkle me with hyssop, and I shall be cleansed; Thou shalt wash
me, and I shall be whiter than snow. Thou shalt make me to hear joy and gladness; my bones, which
have been humbled, shall exult. Turn away Thy face from my sins, and blot out all mine iniquities.
Create in me a clean heart, O God, and renew a right spirit within me.\footnote{Literally, ``in my inwards.'' }Cast me not away from
Thy presence, and take not Thy Holy Spirit from me. Restore to me the joy of Thy salvation, and
establish me by Thy governing Spirit. I will teach transgressors Thy ways, and the ungodly shall
be converted unto Thee. Deliver me from blood-guiltiness,\footnote{Literally, ``bloods.''}O God, the God of my salvation: my
tongue shall exult in Thy righteousness. O Lord, Thou shalt open my mouth, and my lips shall show
forth Thy praise. For if Thou hadst desired sacrifice, I would have given it; Thou wilt not delight
in burnt-offerings. The sacrifice [acceptable] to God is a bruised spirit; a broken and a contrite heart
God will not despise.''\footnote{Ps. li.  1 -  17.}

\sectionnonum[Chapter XIX]{19}{Imitating these examples, let us seek after peace.}
Thus the humility and godly submission of so great and illustrious men have rendered not only
us, but also all the generations before us, better; even as many as have received His oracles in fear
and truth. Wherefore, having so many great and glorious examples set before us, let us turn again
to the practice of that peace which from the beginning was the mark set before us;\footnote{Literally, ``Becoming partakers of many great and glorious deeds, let us return to the aim of peace delivered to us from the beginning.'' Comp. Heb. xii.  1.  }and let us look
stedfastly to the Father and Creator of the universe, and cleave to His mighty and surpassingly great
gifts and benefactions of peace. Let us contemplate Him with our understanding, and look with the
eyes of our soul to His long-suffering will. Let us reflect how free from wrath He is towards all
His creation.

\sectionnonum[Chapter XX]{20}{The peace and harmony of the universe.}
The heavens, revolving under His government, are subject to Him in peace. Day and night run
the course appointed by Him, in no wise hindering each other. The sun and moon, with the companies
of the stars, roll on in harmony according to His command, within their prescribed limits, and
without any deviation. The fruitful earth, according to His will, brings forth food in abundance, at
the proper seasons, for man and beast and all the living beings upon it, never hesitating, nor changing
any of the ordinances which He has fixed. The unsearchable places of abysses, and the indescribable
arrangements of the lower world, are restrained by the same laws. The vast unmeasurable sea,
gathered together by His working into various basins,\footnote{Or, ``collections.'' }never passes beyond the bounds placed
around it, but does as He has commanded. For He said, ``Thus far shalt thou come, and thy waves
shall be broken within thee.''\footnote{Job xxxviii.  11.  }The ocean, impassable to man, and the worlds beyond it, are regulated
by the same enactments of the Lord. The seasons of spring, summer, autumn, and winter, peacefully
give place to one another. The winds in their several quarters\footnote{Or, ``stations.`` }fulfil, at the proper time, their service
without hindrance. The ever-flowing fountains, formed both for enjoyment and health, furnish
without fail their breasts for the life of men. The very smallest of living beings meet together in
peace and concord. All these the great Creator and Lord of all has appointed to exist in peace and
harmony; while He does good to all, but most abundantly to us who have fled for refuge to His
compassions through Jesus Christ our Lord, to whom be glory and majesty for ever and ever. Amen.

\sectionnonum[Chapter XXI]{21}{Let us obey God, and not the authors of sedition.}
Take heed, beloved, lest His many kindnesses lead to the condemnation of us all. [For thus it
must be] unless we walk worthy of Him, and with one mind do those things which are good and
well-pleasing in His sight. For [the Scripture] saith in a certain place, ``The Spirit of the Lord is a
candle searching the secret parts of the belly.''\footnote{Prov. xx.  27.  }Let us reflect how near He is, and that none of the
thoughts or reasonings in which we engage are hid from Him. It is right, therefore, that we should
not leave the post which His will has assigned us. Let us rather offend those men who are foolish,
and inconsiderate, and lifted up, and who glory in the pride of their speech, than [offend] God. Let
us reverence the Lord Jesus Christ, whose blood was given for us; let us esteem those who have
the rule over us;\footnote{Comp. Heb. xiii.  17; 1 Thess. v.  12, 13.}let us honour the aged\footnote{Or, ``the presbyters.''}among us; let us train up the young men in the fear of
God; let us direct our wives to that which is good. Let them exhibit the lovely habit of purity [in
all their conduct]; let them show forth the sincere disposition of meekness; let them make manifest
the command which they have of their tongue, by their manner\footnote{Some read, ``by their silence.''}of speaking; let them display their
love, not by preferring\footnote{Comp.  1 Tim. v.  21.}one to another, but by showing equal affection to all that piously fear God.
Let your children be partakers of true Christian training; let them learn of how great avail humility
is with God - how much the spirit of pure affection can prevail with Him - how excellent and great
His fear is, and how it saves all those who walk in\footnote{Some translate, ``who turn to Him.''}it with a pure mind. For He is a Searcher of
the thoughts and desires [of the heart]: His breath is in us; and when He pleases, He will take it
away.


\sectionnonum[Chapter XXII]{22}{These exhortations are confirmed by the Christian faith, which proclaims
the misery of sinful conduct.}
Now the faith which is in Christ confirms all these [admonitions]. For He Himself by the Holy
Ghost thus addresses us: ``Come, ye children, hearken unto Me; I will teach you the fear of the
Lord. What man is he that desireth life, and loveth to see good days? Keep thy tongue from evil,
and thy lips from speaking guile. Depart from evil, and do good; seek peace, and pursue it. The
eyes of the Lord are upon the righteous, and His ears are [open] unto their prayers. The face of the
Lord is against them that do evil, to cut off the remembrance of them from the earth. The righteous
cried, and the Lord heard him, and delivered him out of all his troubles.''\footnote{Ps. xxxiv.  11 -  17.  }``Many are the stripes
[appointed for] the wicked; but mercy shall compass those about who hope in the Lord.''\footnote{Ps. xxxii.  10.  }

\sectionnonum[Chapter XXIII]{23}{Be humble, and believe that Christ will come again.}
The all-merciful and beneficent Father has bowels [of compassion] towards those that fear Him,
and kindly and lovingly bestows His favours upon those who come to Him with a simple mind.
Wherefore let us not be double-minded; neither let our soul be lifted\footnote{Or, as some render, ``neither let us have any doubt of.'' }up on account of His
exceedingly great and glorious gifts. Far from us be that which is written, ``Wretched are they who
are of a double mind, and of a doubting heart; who say, These things we have heard even in the
times of our fathers; but, behold, we have grown old, and none of them has happened unto us.''\footnote{Some regard these words as taken from an apocryphal book, others as derived from a fusion of Jas. i.  8 and 2 Pet. iii.  3, 4. }
Ye foolish ones! compare yourselves to a tree: take [for instance] the vine. First of all, it sheds its
leaves, then it buds, next it puts forth leaves, and then it flowers; after that comes the sour grape,
and then follows the ripened fruit. Ye perceive how in a little time the fruit of a tree comes to
maturity. Of a truth, soon and suddenly shall His will be accomplished, as the Scripture also bears
witness, saying, ``Speedily will He come, and will not tarry;''\footnote{Hab. ii.  3; Heb. x.  37.  }and, ``The Lord shall suddenly come
to His temple, even the Holy One, for whom ye look.''\footnote{Mal. iii.  1.  }

\sectionnonum[Chapter XXIV]{24}{God continually shows us in nature that there will be a resurrection.}
Let us consider, beloved, how the Lord continually proves to us that there shall be a future
resurrection, of which He has rendered the Lord Jesus Christ the first-fruits\footnote{Comp.  1 Cor. xv.  20; Col. i.  18.  }by raising Him from
the dead. Let us contemplate, beloved, the resurrection which is at all times taking place. Day and
night declare to us a resurrection. The night sinks to sleep, and the day arises; the day [again]
departs, and the night comes on. Let us behold the fruits [of the earth], how the sowing of grain
takes place. The sower\footnote{Comp. Luke viii.  5.  }goes forth, and casts it into the ground; and the seed being thus scattered,
though dry and naked when it fell upon the earth, is gradually dissolved. Then out of its dissolution
the mighty power of the providence of the Lord raises it up again, and from one seed many arise
and bring forth fruit.

\sectionnonum[Chapter XXV]{25}{The phoenix an emblem of our resurrection.}
Let us consider that wonderful sign [of the resurrection] which takes place in Eastern lands,
that is, in Arabia and the countries round about. There is a certain bird which is called a phoenix.
This is the only one of its kind, and lives five hundred years. And when the time of its dissolution
draws near that it must die, it builds itself a nest of frankincense, and myrrh, and other spices, into
which, when the time is fulfilled, it enters and dies. But as the flesh decays a certain kind of worm
is produced, which, being nourished by the juices of the dead bird, brings forth feathers. Then,
when it has acquired strength, it takes up that nest in which are the bones of its parent, and bearing
these it passes from the land of Arabia into Egypt, to the city called Heliopolis. And, in open day,
flying in the sight of all men, it places them on the altar of the sun, and having done this, hastens
back to its former abode. The priests then inspect the registers of the dates, and find that it has
returned exactly as the five hundredth year was completed.\footnote{This fable respecting the phoenix is mentioned by Herodotus (ii.  73) and by Pliny (Nat. Hist., x.  2.) and is used as above by Tertullian (De Resurr., § 13) and by others of the Fathers.}

\sectionnonum[Chapter XXVI]{26}{We shall rise again, then, as the Scripture also testifies.}
Do we then deem it any great and wonderful thing for the Maker of all things to raise up again
those that have piously served Him in the assurance of a good faith, when even by a bird He shows
us the mightiness of His power to fulfil His promise?\footnote{Literally, ``the mightiness of His promise.''}For [the Scripture] saith in a certain place,
``Thou shalt raise me up, and I shall confess unto Thee;''\footnote{Ps. xxviii.  7, or some apocryphal book.  }and again, ``I laid me down, and slept;
I awaked, because Thou art with me;''\footnote{Comp. Ps. iii.  6.  }and again, Job says, ``Thou shalt raise up this flesh of mine,
which has suffered all these things.''\footnote{Job xix.  25, 26.  }

\sectionnonum[Chapter XXVII]{27}{In the hope of the resurrection, let us cleave to the omnipotent and
omniscient God.}
Having then this hope, let our souls be bound to Him who is faithful in His promises, and just
in His judgments. He who has commanded us not to lie, shall much more Himself not lie; for
nothing is impossible with God, except to lie.\footnote{Comp. Tit. i.  2; Heb. vi.  18.  }Let His faith therefore be stirred up again within
us, and let us consider that all things are nigh unto Him. By the word of His might\footnote{Or, ``majesty.''}He established
all things, and by His word He can overthrow them. ``Who shall say unto Him, What hast thou
done? or, Who shall resist the power of His strength?''\footnote{Wisdom xii.  12, Wisdom xi.  22.  }When and as He pleases He will do all
things, and none of the things determined by Him shall pass away.\footnote{Comp. Matt. xxiv.  35.  }All things are open before
Him, and nothing can be hidden from His counsel. ``The heavens\footnote{Literally, ``If the heavens,'' etc }declare the glory of God, and
the firmament showeth His handy-work. Day unto day uttereth speech, and night unto night showeth
knowledge. And there are no words or speeches of which the voices are not heard.''\footnote{Ps. xix.  1 - 3.}

\sectionnonum[Chapter XXVIII]{28}{God sees all things: therefore let us avoid transgression.}
Since then all things are seen and heard [by God], let us fear Him, and forsake those wicked
works which proceed from evil desires;\footnote{Literally, ``abominable lusts of evil deeds.''}so that, through His mercy, we may be protected from
the judgments to come. For whither can any of us flee from His mighty hand? Or what world will
receive any of those who run away from Him? For the Scripture saith in a certain place, ``Whither
shall I go, and where shall I be hid from Thy presence? If I ascend into heaven, Thou art there; if
I go away even to the uttermost parts of the earth, there is Thy right hand; if I make my bed in the
abyss, there is Thy Spirit.''\footnote{Ps. cxxxix.  7 -  10.  }Whither, then, shall any one go, or where shall he escape from Him
who comprehends all things?

\sectionnonum[Chapter XXIX]{29}{Let us also draw near to God in purity of heart.}
Let us then draw near to Him with holiness of spirit, lifting up pure and undefiled hands unto
Him, loving our gracious and merciful Father, who has made us partakers in the blessings of His
elect.\footnote{Literally ``has made us to Himself a part of election.''}For thus it is written, ``When the Most High divided the nations, when He scattered\footnote{Literally, ``sowed abroad.''}the
sons of Adam, He fixed the bounds of the nations according to the number of the angels of God.
His people Jacob became the portion of the Lord, and Israel the lot of His inheritance.''\footnote{Deut. xxxii.  8, 9.  }And in
another place [the Scripture] saith, ``Behold, the Lord taketh unto Himself a nation out of the midst
of the nations, as a man takes the first-fruits of his threshing-floor; and from that nation shall come
forth the Most Holy.''\footnote{Formed apparently from Num. xviii.  27 and 2 Chron. xxxi.  14. Literally, the closing words are, ``the holy of holies.''}

\sectionnonum[Chapter XXX]{30}{Let us do those things that please God, and flee from those He hates, that
we may be blessed.}
Seeing, therefore, that we are the portion of the Holy One, let us do all those things which
pertain to holiness, avoiding all evil-speaking, all abominable and impure embraces, together with
all drunkenness, seeking after change,\footnote{Some translate, ``youthful lusts.''}all abominable lusts, detestable adultery, and execrable
pride. ``For God,'' saith [the Scripture], ``resisteth the proud, but giveth grace to the humble.''\footnote{Prov. iii.  34; Jas. iv.  6; 1 Pet. v.  5.  }Let
us cleave, then, to those to whom grace has been given by God. Let us clothe ourselves with concord
and humility, ever exercising self-control, standing far off from all whispering and evil-speaking,
being justified by our works, and not our words. For [the Scripture] saith, ``He that speaketh much,
shall also hear much in answer. And does he that is ready in speech deem himself righteous? Blessed
is he that is born of woman, who liveth but a short time: be not given to much speaking.''\footnote{Job xi.  2, 3. The translation is doubtful. [But see Septuagint.] }Let our
praise be in God, and not of ourselves; for God hateth those that commend themselves. Let testimony
to our good deeds be borne by others, as it was in the case of our righteous forefathers. Boldness,
and arrogance, and audacity belong to those that are accursed of God; but moderation, humility,
and meekness to such as are blessed by Him.

\sectionnonum[Chapter XXXI]{31}{Let us see by what means we may obtain the divine blessing.}
Let us cleave then to His blessing, and consider what are the means\footnote{Literally, ``what are the ways of His blessing.''}of possessing it. Let us
think\footnote{Literally, ``unroll.''}over the things which have taken place from the beginning. For what reason was our father
Abraham blessed? was it not because he wrought righteousness and truth through faith?\footnote{Comp. Jas. ii.  21.  }Isaac,
with perfect confidence, as if knowing what was to happen,\footnote{Some translate, ``knowing what was to come.''}cheerfully yielded himself as a
sacrifice.\footnote{Gen. xxii.  }Jacob, through reason\footnote{So Jacobson: Wotton reads, ``fleeing from his brother.''}of his brother, went forth with humility from his own land,
and came to Laban and served him; and there was given to him the sceptre of the twelve tribes of
Israel.

\sectionnonum[Chapter XXXII]{32}{We are justified not by our own works, but by faith.}
Whosoever will candidly consider each particular, will recognise the greatness of the gifts which
were given by him.\footnote{The meaning is here very doubtful. Some translate ``the gifts which were given to Jacob by Him,'' i.e., God.  }For from him\footnote{MS. \textgreek{ant\~{w}n}, referring to the gifts: we have followed the emendation αὐτοῦ, adopted by most editors. Some refer the word to God, and not Jacob.  }have sprung the priests and all the Levites who minister at
the altar of God. From him also [was descended] our Lord Jesus Christ according to the flesh.\footnote{Comp. Rom. ix.  5.  }From him [arose] kings, princes, and rulers of the race of Judah. Nor are his other tribes in small
glory, inasmuch as God had promised, ``Thy seed shall be as the stars of heaven.''\footnote{Gen. xxii.  17, Gen. xxviii.  4.  }All these,
therefore, were highly honoured, and made great, not for their own sake, or for their own works,
or for the righteousness which they wrought, but through the operation of His will. And we, too,
being called by His will in Christ Jesus, are not justified by ourselves, nor by our own wisdom, or
understanding, or godliness, or works which we have wrought in holiness of heart; but by that faith
through which, from the beginning, Almighty God has justified all men; to whom be glory for ever
and ever. Amen.

\sectionnonum[Chapter XXXIII]{33}{But let us not give up the practice of good works and love. God Himself
is an example to us of good works.}
What shall we do, then, brethren? Shall we become slothful in well-doing, and cease from the
practice of love? God forbid that any such course should be followed by us! But rather let us hasten
with all energy and readiness of mind to perform every good work. For the Creator and Lord of all
Himself rejoices in His works. For by His infinitely great power He established the heavens, and
by His incomprehensible wisdom He adorned them. He also divided the earth from the water which
surrounds it, and fixed it upon the immoveable foundation of His own will. The animals also which
are upon it He commanded by His own word\footnote{Or, ``commandment.''}into existence. So likewise, when He had formed
the sea, and the living creatures which are in it, He enclosed them [within their proper bounds] by
His own power. Above all,\footnote{Or, ``in addition to all.''}with His holy and undefiled hands He formed man, the most excellent
[of His creatures], and truly great through the understanding given him -  the express likeness of
His own image. For thus says God: ``Let us make man in Our image, and after Our likeness. So
God made man; male and female He created them.''\footnote{Gen. i.  26, 27.  }Having thus finished all these things, He
approved them, and blessed them, and said, ``Increase and multiply.''\footnote{Gen. i.  28.  }We see,\footnote{Or, ``let us consider.''}then, how all
righteous men have been adorned with good works, and how the Lord Himself, adorning Himself
with His works, rejoiced. Having therefore such an example, let us without delay accede to His
will, and let us work the work of righteousness with our whole strength.

\sectionnonum[Chapter XXXIV]{34}{Great is the reward of good works with God. Joined together in harmony,
let us implore that reward from Him.} The good servant\footnote{Or, ``labourer.''}receives the bread of his labour with confidence; the lazy and slothful cannot
look his employer in the face. It is requisite, therefore, that we be prompt in the practice of
well-doing; for of Him are all things. And thus He forewarns us: ``Behold, the Lord [cometh], and
His reward is before His face, to render to every man according to his work.''\footnote{Isa. xl.  10, Isa. lxii.  11; Rev. xxii.  12.  }He exhorts us,
therefore, with our whole heart to attend to this,\footnote{The text here seems to be corrupt. Some translate, ``He warns us with all His heart to this end, that,'' etc.  }that we be not lazy or slothful in any good work.
Let our boasting and our confidence be in Him. Let us submit ourselves to His will. Let us consider
the whole multitude of His angels, how they stand ever ready to minister to His will. For the Scripture
saith, ``Ten thousand times ten thousand stood around Him, and thousands of thousands ministered
unto Him,\footnote{Dan. vii.  10.  }and cried, Holy, holy, holy, [is] the Lord of Sabaoth; the whole creation is full of His
glory.''\footnote{Isa. vi.  3.  }And let us therefore, conscientiously gathering together in harmony, cry to Him earnestly,
as with one mouth, that we may be made partakers of His great and glorious promises. For [the
Scripture] saith, ``Eye hath not seen, nor ear heard, neither have entered into the heart of man, the
things which He hath prepared for them that wait for Him.''\footnote{1 Cor. ii.  9.  }

\sectionnonum[Chapter XXXV]{35}{Immense is this reward.} How shall we obtain it?
How blessed and wonderful, beloved, are the gifts of God! Life in immortality, splendour in
righteousness, truth in perfect confidence,\footnote{Some translate, ``in liberty.'' }faith in assurance, self-control in holiness! And all
these fall under the cognizance of our understandings [now]; what then shall those things be which
are prepared for such as wait for Him? The Creator and Father of all worlds,\footnote{Or, ``of the ages.'' }the Most Holy, alone
knows their amount and their beauty. Let us therefore earnestly strive to be found in the number
of those that wait for Him, in order that we may share in His promised gifts. But how, beloved,
shall this be done? If our understanding be fixed by faith towards God; if we earnestly seek the
things which are pleasing and acceptable to Him; if we do the things which are in harmony with
His blameless will; and if we follow the way of truth, casting away from us all unrighteousness
and iniquity, along with all covetousness, strife, evil practices, deceit, whispering, and evil-speaking,
all hatred of God, pride and haughtiness, vainglory and ambition.\footnote{The reading is doubtful: some have ἀφιλοξενίαν, ``want of a hospitable spirit.'' [So Jacobson.]}For they that do such things
are hateful to God; and not only they that do them, but also those that take pleasure in them that
do them.\footnote{Rom. i.  32.  }For the Scripture saith, ``But to the sinner God said, Wherefore dost thou declare my
statutes, and take my covenant into thy mouth, seeing thou hatest instruction, and castest my words
behind thee? When thou sawest a thief, thou consentedst with\footnote{Literally, ``didst run with.'' }him, and didst make thy portion
with adulterers. Thy mouth has abounded with wickedness, and thy tongue contrived\footnote{Literally, ``didst weave.'' }deceit. Thou
sittest, and speakest against thy brother; thou slanderest\footnote{Or, ``layest a snare for.'' }thine own mother’s son. These things
thou hast done, and I kept silence; thou thoughtest, wicked one, that I should be like to thyself. But
I will reprove thee, and set thyself before thee. Consider now these things, ye that forget God, lest
He tear you in pieces, like a lion, and there be none to deliver. The sacrifice of praise will glorify
Me, and a way is there by which I will show him the salvation of God.''\footnote{Ps. l.  16 -  23. The reader will observe how the Septuagint followed by Clement differs from the Hebrew.}

\sectionnonum[Chapter XXXVI]{36}{All blessings are given to us through Christ.}
This is the way, beloved, in which we find our Saviour,\footnote{Literally, ``that which saves us.'' }even Jesus Christ, the High Priest of
all our offerings, the defender and helper of our infirmity. By Him we look up to the heights of
heaven. By Him we behold, as in a glass, His immaculate and most excellent visage. By Him are
the eyes of our hearts opened. By Him our foolish and darkened understanding blossoms\footnote{Or, ``rejoices to behold.'' }up anew
towards His marvellous light. By Him the Lord has willed that we should taste of immortal
knowledge,\footnote{Or, ``knowledge of immortality.'' }``who, being the brightness of His majesty, is by so much greater than the angels, as
He hath by inheritance obtained a more excellent name than they.''\footnote{Heb. i.  3, 4.  }For it is thus written, ``Who
maketh His angels spirits, and His ministers a flame of fire.''\footnote{Ps. civ.  4; Heb. i.  7.  }But concerning His Son\footnote{Some render, ``to the Son.''}the Lord
spoke thus: ``Thou art my Son, to-day have I begotten Thee. Ask of Me, and I will give Thee the
heathen for Thine inheritance, and the uttermost parts of the earth for Thy possession.''\footnote{Ps. ii.  7, 8; Heb. i.  5.  }And again
He saith to Him, ``Sit Thou at My right hand, until I make Thine enemies Thy footstool.''\footnote{Ps. cx.  1; Heb. i.  13.}But
who are His enemies? All the wicked, and those who set themselves to oppose the will of God.\footnote{Some read, ``who oppose their own will to that of God.''}

\sectionnonum[Chapter XXXVII]{37}{Christ is our leader, and we His soldiers.}
Let us then, men and brethren, with all energy act the part of soldiers, in accordance with His
holy commandments. Let us consider those who serve under our generals, with what order,
obedience, and submissiveness they perform the things which are commanded them. All are not
prefects, nor commanders of a thousand, nor of a hundred, nor of fifty, nor the like, but each one
in his own rank performs the things commanded by the king and the generals. The great cannot
subsist without the small, nor the small without the great. There is a kind of mixture in all things,
and thence arises mutual advantage.\footnote{Literally, ``in these there is use.''}Let us take our body for an example.\footnote{1 Cor. xii.  12, etc.  }The head is nothing
without the feet, and the feet are nothing without the head; yea, the very smallest members of our
body are necessary and useful to the whole body. But all work\footnote{Literally, ``all breathe together.''}harmoniously together, and are
under one common rule\footnote{Literally, ``use one subjection.''}for the preservation of the whole body.

\sectionnonum[Chapter XXXVIII]{38}{Let the members of the Church submit themselves, and no one exalt
himself above another.}
Let our whole body, then, be preserved in, Christ Jesus; and let every one be subject to his
neighbour, according to the special gift\footnote{Literally, ``according as he has been placed in his charism.''}bestowed upon him. Let the strong not despise the weak,
and let the weak show respect unto the strong. Let the rich man provide for the wants of the poor;
and let the poor man bless God, because He hath given him one by whom his need may be supplied.
Let the wise man display his wisdom, not by [mere] words, but through good deeds. Let the humble
not bear testimony to himself, but leave witness to be borne to him by another.\footnote{Comp. Prov. xxvii.  2.  }Let him that is
pure in the flesh not grow proud\footnote{The MS. is here slightly torn, and we are left to conjecture.}of it, and boast, knowing that it was another who bestowed on
him the gift of continence. Let us consider, then, brethren, of what matter we were made, - who
and what manner of beings we came into the world, as it were out of a sepulchre, and from utter
darkness.\footnote{Comp. Ps. cxxxix.  15.  }He who made us and fashioned us, having prepared His bountiful gifts for us before
we were born, introduced us into His world. Since, therefore, we receive all these things from Him,
we ought for everything to give Him thanks; to whom be glory for ever and ever. Amen.

\sectionnonum[Chapter XXXIX]{39}{There is no reason for self-conceit.}
Foolish and inconsiderate men, who have neither wisdom\footnote{Literally, ``and silly and uninstructed.''}nor instruction, mock and deride
us, being eager to exalt themselves in their own conceits. For what can a mortal man do? or what
strength is there in one made out of the dust? For it is written, ``There was no shape before mine
eyes, only I heard a sound,\footnote{Literally, ``a breath.''}and a voice [saying], What then? Shall a man be pure before the Lord?
or shall such an one be [counted] blameless in his deeds, seeing He does not confide in His servants,
and has charged\footnote{Or, ``has perceived.''}even His angels with perversity? The heaven is not clean in His sight: how much
less they that dwell in houses of clay, of which also we ourselves were made! He smote them as a
moth; and from morning even until evening they endure not. Because they could furnish no assistance
to themselves, they perished. He breathed upon them, and they died, because they had no wisdom.
But call now, if any one will answer thee, or if thou wilt look to any of the holy angels; for wrath
destroys the foolish man, and envy killeth him that is in error. I have seen the foolish taking root,
but their habitation was presently consumed. Let their sons be far from safety; let them be despised\footnote{Some render, ``they perished at the gates.`` }before the gates of those less than themselves, and there shall be none to deliver. For what was
prepared for them, the righteous shall eat; and they shall not be delivered from evil.''\footnote{Job iv.  16 -  18, Job xv.  15, Job iv.  19 -  21, Job v.  1 -  5.}

\sectionnonum[Chapter XL]{40}{Let us preserve in the Church the order appointed by God.}
These things therefore being manifest to us, and since we look into the depths of the divine
knowledge, it behoves us to do all things in [their proper] order, which the Lord has commanded
us to perform at stated times.\footnote{Some join \textgreek{kat\'{a} kairo\`{u}z tetagm\'{e}nouz}, ``at stated times.'' to the next sentence. [ 1 Cor. xvi.  1, 2.] }He has enjoined offerings [to be presented] and service to be
performed [to Him], and that not thoughtlessly or irregularly, but at the appointed times and hours.
Where and by whom He desires these things to be done, He Himself has fixed by His own supreme
will, in order that all things being piously done according to His good pleasure, may be acceptable
unto Him.\footnote{Literally, ``to His will.'' [Comp. Rom. xv.  15, 16, Greek.] }Those, therefore, who present their offerings at the appointed times, are accepted and
blessed; for inasmuch as they follow the laws of the Lord, they sin not. For his own peculiar services
are assigned to the high priest, and their own proper place is prescribed to the priests, and their own
special ministrations devolve on the Levites. The layman is bound by the laws that pertain to laymen.

\sectionnonum[Chapter XLI]{41}{Continuation of the same subject.}
Let every one of you, brethren, give thanks to God in his own order, living in all good conscience,
with becoming gravity, and not going beyond the rule of the ministry prescribed to him. Not in
every place, brethren, are the daily sacrifices offered, or the peace-offerings, or the sin-offerings
and the trespass-offerings, but in Jerusalem only. And even there they are not offered in any place,
but only at the altar before the temple, that which is offered being first carefully examined by the
high priest and the ministers already mentioned. Those, therefore, who do anything beyond that
which is agreeable to His will, are punished with death. Ye see,\footnote{Or, ``consider.'' [This chapter has been cited to prove the earlier date for this Epistle. But the reference to Jerusalem may be an ideal present.] }brethren, that the greater the
knowledge that has been vouchsafed to us, the greater also is the danger to which we are exposed.

\sectionnonum[Chapter XLII]{42}{The order of ministers in the Church.}
The apostles have preached the Gospel to us from\footnote{Or, ``by the command of.'' }the Lord Jesus Christ; Jesus Christ [has
done so] from\footnote{Or, ``by the command of.'' }God. Christ therefore was sent forth by God, and the apostles by Christ. Both these
appointments,\footnote{Literally, ``both things were done.'' }then, were made in an orderly way, according to the will of God. Having therefore
received their orders, and being fully assured by the resurrection of our Lord Jesus Christ, and
established\footnote{Or, ``confirmed by.''}in the word of God, with full assurance of the Holy Ghost, they went forth proclaiming
that the kingdom of God was at hand. And thus preaching through countries and cities, they appointed
the first-fruits [of their labours], having first proved them by the Spirit,\footnote{Or, ``having tested them in spirit.'' }to be bishops and deacons
of those who should afterwards believe. Nor was this any new thing, since indeed many ages before
it was written concerning bishops and deacons. For thus saith the Scripture in a certain place, ``I
will appoint their bishops\footnote{Or, ``overseers.'' }in righteousness, and their deacons\footnote{Or, ``servants.''}in faith.''\footnote{Isa. lx.  17, Sept.; but the text is here altered by Clement. The LXX. have ``I will give thy rulers in peace, and thy overseers in righteousness.'' }

\sectionnonum[Chapter XLIII]{43}{Moses of old stilled the contention which arose concerning the priestly
dignity.}
And what wonder is it if those in Christ who were entrusted with such a duty by God, appointed
those [ministers] before mentioned, when the blessed Moses also, ``a faithful servant in all his
house,''\footnote{Num. xii.  7; Heb. iii.  5.  }noted down in the sacred books all the injunctions which were given him, and when the
other prophets also followed him, bearing witness with one consent to the ordinances which he had
appointed? For, when rivalry arose concerning the priesthood, and the tribes were contending among
themselves as to which of them should be adorned with that glorious title, he commanded the twelve
princes of the tribes to bring him their rods, each one being inscribed with the name\footnote{Literally, ``every tribe being written according to its name.''}of the tribe.
And he took them and bound them [together], and sealed them with the rings of the princes of the
tribes, and laid them up in the tabernacle of witness on the table of God. And having shut the doors
of the tabernacle, he sealed the keys, as he had done the rods, and said to them, Men and brethren,
the tribe whose rod shall blossom has God chosen to fulfil the office of the priesthood, and to
minister unto Him. And when the morning was come, he assembled all Israel, six hundred thousand
men, and showed the seals to the princes of the tribes, and opened the tabernacle of witness, and
brought forth the rods. And the rod of Aaron was found not only to have blossomed, but to bear
fruit upon it.\footnote{See Num. xvii.}What think ye, beloved? Did not Moses know beforehand that this would happen?
Undoubtedly he knew; but he acted thus, that there might be no sedition in Israel, and that the
name of the true and only God might be glorified; to whom be glory for ever and ever. Amen.

\sectionnonum[Chapter XLIV]{44}{The ordinances of the apostles, that there might be no contention respecting
the priestly office.}
Our apostles also knew, through our Lord Jesus Christ, and there would be strife on account of
the office\footnote{Literally, ``on account of the title of the oversight.'' Some understand this to mean, ``in regard to the dignity of the episcopate;'' and others simply, ``on account of the oversight.'' }of the episcopate. For this reason, therefore, inasmuch as they had obtained a perfect
fore-knowledge of this, they appointed those [ministers] already mentioned, and afterwards gave
instructions,\footnote{The meaning of this passage is much controverted. Some render, ``left a list of other approved persons;'' while others translate the unusual word \textgreek{epinom\'{h}}, which causes the difficulty, by ``testamentary direction,'' and many others deem the text corrupt. We have given what seems the simplest version of the text as it stands. [Comp. the versions of Wake, Chevallier, and others.] }that when these should fall asleep, other approved men should succeed them in their
ministry. We are of opinion, therefore, that those appointed by them,\footnote{i.e., the apostles.  }or afterwards by other
eminent men, with the consent of the whole Church, and who have blamelessly served the flock
of Christ in a humble, peaceable, and disinterested spirit, and have for a long time possessed the
good opinion of all, cannot be justly dismissed from the ministry. For our sin will not be small, if
we eject from the episcopate\footnote{Or, ``oversight.''}those who have blamelessly and holily fulfilled its duties.\footnote{Literally, ``presented the offerings.''}Blessed
are those presbyters who, having finished their course before now, have obtained a fruitful and
perfect departure [from this world]; for they have no fear lest any one deprive them of the place
now appointed them. But we see that ye have removed some men of excellent behaviour from the
ministry, which they fulfilled blamelessly and with honour.

\sectionnonum[Chapter XLV]{45}{It is the part of the wicked to vex the righteous.}
Ye are fond of contention, brethren, and full of zeal about things which do not pertain to
salvation. Look carefully into the Scriptures, which are the true utterances of the Holy Spirit.
Observe\footnote{Or, ``Ye perceive.''}that nothing of an unjust or counterfeit character is written in them. There\footnote{Or, ``For.''}you will
not find that the righteous were cast off by men who themselves were holy. The righteous were
indeed persecuted, but only by the wicked. They were cast into prison, but only by the unholy; they
were stoned, but only by transgressors; they were slain, but only by the accursed, and such as had
conceived an unrighteous envy against them. Exposed to such sufferings, they endured them
gloriously. For what shall we say, brethren? Was Daniel\footnote{Dan. vi.  16.}cast into the den of lions by such as
feared God? Were Ananias, and Azarias, and Mishaël shut up in a furnace\footnote{Dan. iii.  20.}of fire by those who
observed\footnote{Literally, ``worshipped.''}the great and glorious worship of the Most High? Far from us be such a thought! Who,
then, were they that did such things? The hateful, and those full of all wickedness, were roused to
such a pitch of fury, that they inflicted torture on those who served God with a holy and blameless
purpose [of heart], not knowing that the Most High is the Defender and Protector of all such as
with a pure conscience venerate\footnote{Literally, ``serve.''}His all-excellent name; to whom be glory for ever and ever.
Amen. But they who with confidence endured [these things] are now heirs of glory and honour,
and have been exalted and made illustrious\footnote{Or, ``lifted up.''}by God in their memorial for ever and ever. Amen.

\sectionnonum[Chapter XLVI]{46}{Let us cleave to the righteous: your strife is pernicious.}
Such examples, therefore, brethren, it is right that we should follow;\footnote{Literally, ``To such examples it is right that we should cleave.''}since it is written, ``Cleave
to the holy, for those that cleave to them shall [themselves] be made holy.''\footnote{Not found in Scripture.  }And again, in another
place, [the Scripture] saith, ``With a harmless man thou shalt prove\footnote{Literally, ``be.''}thyself harmless, and with an
elect man thou shalt be elect, and with a perverse man thou shalt show\footnote{Or, ``thou wilt overthrow.''}thyself perverse.''\footnote{Ps. xviii.  25, 26.  }Let
us cleave, therefore, to the innocent and righteous, since these are the elect of God. Why are there
strifes, and tumults, and divisions, and schisms, and wars\footnote{Or, ``war.'' Comp. Jas. iv.  1.  }among you? Have we not [all] one God
and one Christ? Is there not one Spirit of grace poured out upon us? And have we not one calling
in Christ?\footnote{Comp. Eph. iv.  4 -  6.}Why do we divide and tear to pieces the members of Christ, and raise up strife against
our own body, and have reached such a height of madness as to forget that ``we are members one
of another?''\footnote{Rom. xii.  5.  }Remember the words of our Lord Jesus Christ, how\footnote{This clause is wanting in the text.  }He said, ``Woe to that man
[by whom\footnote{This clause is wanting in the text.  }offences come]! It were better for him that he had never been born, than that he should
cast a stumbling-block before one of my elect. Yea, it were better for him that a millstone should

be hung about [his neck], and he should be sunk in the depths of the sea, than that he should cast
a stumbling-block before one of my little ones.''\footnote{Comp. Matt. xviii.  6, Matt. xxvi.  24; Mark ix.  42; Luke xvii.  2.  }Your schism has subverted [the faith of] many,
has discouraged many, has given rise to doubt in many, and has caused grief to us all. And still
your sedition continueth.

\sectionnonum[Chapter XLVII]{47}{Your recent discord is worse than the former which took place in the times
of Paul.}
Take up the epistle of the blessed Apostle Paul. What did he write to you at the time when the
Gospel first began to be preached?\footnote{Literally, ``in the beginning of the Gospel.'' [Comp. Phil. iv.  15.] }Truly, under the inspiration\footnote{Or, ``spiritually.''}of the Spirit, he wrote to you
concerning himself, and Cephas, and Apollos,\footnote{1 Cor. iii.  13, etc.  }because even then parties\footnote{Or, ``inclinations for one above another.''}had been formed
among you. But that inclination for one above another entailed less guilt upon you, inasmuch as
your partialities were then shown towards apostles, already of high reputation, and towards a man
whom they had approved. But now reflect who those are that have perverted you, and lessened the
renown of your far-famed brotherly love. It is disgraceful, beloved, yea, highly disgraceful, and
unworthy of your Christian profession,\footnote{Literally, ``of conduct in Christ.''}that such a thing should be heard of as that the most
stedfast and ancient Church of the Corinthians should, on account of one or two persons, engage
in sedition against its presbyters. And this rumour has reached not only us, but those also who are
unconnected\footnote{Or, ``aliens from us,'' i.e., the Gentiles.}with us; so that, through your infatuation, the name of the Lord is blasphemed, while
danger is also brought upon yourselves.

\sectionnonum[Chapter XLVIII]{48}{Let us return to the practice of brotherly love.}
Let us therefore, with all haste, put an end\footnote{Literally ``remove.'' }to this [state of things]; and let us fall down before
the Lord, and beseech Him with tears, that He would mercifully\footnote{Literally, ``becoming merciful.'' }be reconciled to us, and restore
us to our former seemly and holy practice of brotherly love. For [such conduct] is the gate of
righteousness, which is set open for the attainment of life, as it is written, ``Open to me the gates
of righteousness; I will go in by them, and will praise the Lord: this is the gate of the Lord: the
righteous shall enter in by it.''\footnote{Ps. cxviii.  19, 20.  }Although, therefore, many gates have been set open, yet this gate
of righteousness is that gate in Christ by which blessed are all they that have entered in and have
directed their way in holiness and righteousness, doing all things without disorder. Let a man be
faithful: let him be powerful in the utterance of knowledge; let him be wise in judging of words;
let him be pure in all his deeds; yet the more he seems to be superior to others [in these respects],
the more humble-minded ought he to be, and to seek the common good of all, and not merely his
own advantage.

\sectionnonum[Chapter XLIX]{49}{The praise of love.}
Let him who has love in Christ keep the commandments of Christ. Who can describe the
[blessed] bond of the love of God? What man is able to tell the excellence of its beauty, as it ought
to be told? The height to which love exalts is unspeakable. Love unites us to God. Love covers a
multitude of sins.\footnote{Jas. v.  20; 1 Pet. iv.  8.  }Love beareth all things, is long-suffering in all things.\footnote{Comp.  1 Cor. xiii.  4, etc.  }There is nothing base,
nothing arrogant in love. Love admits of no schisms: love gives rise to no seditions: love does all
things in harmony. By love have all the elect of God been made perfect; without love nothing is
well-pleasing to God. In love has the Lord taken us to Himself. On account of the Love he bore us,
Jesus Christ our Lord gave His blood for us by the will of God; His flesh for our flesh, and His soul
for our souls.\footnote{[Comp. Iren{\ae}us, v.  1; also Mathetes, Ep. to Diognetus, cap. ix.]}

\sectionnonum[Chapter L]{50}{Let us pray to be thought worthy of love.}
Ye see, beloved, how great and wonderful a thing is love, and that there is no declaring its
perfection. Who is fit to be found in it, except such as God has vouchsafed to render so? Let us
pray, therefore, and implore of His mercy, that we may live blameless in love, free from all human
partialities for one above another. All the generations from Adam even unto this day have passed
away; but those who, through the grace of God, have been made perfect in love, now possess a
place among the godly, and shall be made manifest at the revelation\footnote{Literally, ``visitation.`` }of the kingdom of Christ.
For it is written, ``Enter into thy secret chambers for a little time, until my wrath and fury pass away;
and I will remember a propitious\footnote{Or, ``good.`` }day, and will raise you up out of your graves.``\footnote{Isa. xxvi.  20.  }Blessed are
we, beloved, if we keep the commandments of God in the harmony of love; that so through love
our sins may be forgiven us. For it is written, ``Blessed are they whose transgressions are forgiven,
and whose sins are covered. Blessed is the man whose sin the Lord will not impute to him, and in
whose mouth there is no guile.``\footnote{Ps. xxxii.  1, 2.  }This blessedness cometh upon those who have been chosen by
God through Jesus Christ our Lord; to whom be glory for ever and ever. Amen.

\sectionnonum[Chapter LI]{51}{Let the partakers in strife acknowledge their sins.}
Let us therefore implore forgiveness for all those transgressions which through any [suggestion]
of the adversary we have committed. And those who have been the leaders of sedition and
disagreement ought to have respect\footnote{Or, ``look to.`` }to the common hope. For such as live in fear and love would
rather that they themselves than their neighbours should be involved in suffering. And they prefer
to bear blame themselves, rather than that the concord which has been well and piously\footnote{Or, ``righteously.`` }handed
down to us should suffer. For it is better that a man should acknowledge his transgressions than
that he should harden his heart, as the hearts of those were hardened who stirred up sedition against
Moses the servant of God, and whose condemnation was made manifest [unto all]. For they went
down alive into Hades, and death swallowed them up.\footnote{Num. xvi.  }Pharaoh with his army and all the princes
of Egypt, and the chariots with their riders, were sunk in the depths of the Red Sea, and perished,\footnote{Ex. xiv.}for no other reason than that their foolish hearts were hardened, after so many signs and wonders
had been wrought in the land of Egypt by Moses the servant of God.


\sectionnonum[Chapter LII]{52}{Such a confession is pleasing to God.}
The Lord, brethren, stands in need of nothing; and He desires nothing of any one, except that
confession be made to Him. For, says the elect David, ``I will confess unto the Lord; and that will
please Him more than a young bullock that hath horns and hoofs. Let the poor see it, and be glad.``\footnote{Ps. lxix.  31, 32.  }And again he saith, ``Offer\footnote{Or, ``sacrifice.`` }unto God the sacrifice of praise, and pay thy vows unto the Most High.
And call upon Me in the day of thy trouble: I will deliver thee, and thou shalt glorify Me.``\footnote{Ps. l.  14, 15.  }For
``the sacrifice of God is a broken spirit.``\footnote{Ps. li.  17.}

\sectionnonum[Chapter LIII]{53}{The love of Moses towards his people.}
Ye understand, beloved, ye understand well the Sacred Scriptures, and ye have looked very
earnestly into the oracles of God. Call then these things to your remembrance. When Moses went
up into the mount, and abode there, with fasting and humiliation, forty days and forty nights, the
Lord said unto him, ``Moses, Moses, get thee down quickly from hence; for thy people whom thou
didst bring out of the land of Egypt have committed iniquity. They have speedily departed from
the way in which I commanded them to walk, and have made to themselves molten images.``\footnote{Ex. xxxii.  7, etc.; Deut. ix.  12, etc.  }And the Lord said unto him, ``I have spoken to thee once and again, saying, I have seen this people,
and, behold, it is a stiff-necked people: let Me destroy them, and blot out their name from under
heaven; and I will make thee a great and wonderful nation, and one much more numerous than
this.``\footnote{Ex. xxxii.  9, etc.  }But Moses said, ``Far be it from Thee, Lord: pardon the sin of this people; else blot me also
out of the book of the living.``\footnote{Ex. xxxii.  32.  }O marvellous\footnote{Or, ``mighty.`` }love! O insuperable perfection! The servant speaks
freely to his Lord, and asks forgiveness for the people, or begs that he himself might perish\footnote{Literally, ``be wiped out.``}along
with them.

\sectionnonum[Chapter LIV]{54}{He who is full of love will incur every loss, that peace may be restored to the
Church.}
Who then among you is noble-minded? who compassionate? who full of love? Let him declare,
``If on my account sedition and disagreement and schisms have arisen, I will depart, I will go away
whithersoever ye desire, and I will do whatever the majority\footnote{Literally, ``the multitude.`` [Clement here puts words into the mouth of the Corinthian presbyters. It has been strangely quoted to strengthen a conjecture that he had humbly preferred Linus and Cletus when first called to preside.] }commands; only let the flock of
Christ live on terms of peace with the presbyters set over it.`` He that acts thus shall procure to
himself great glory in the Lord; and every place will welcome\footnote{Or, ``receive.`` }him. For ``the earth is the Lord’s,
and the fulness thereof.``\footnote{Ps. xxiv.  1; 1 Cor. x.  26, 28.  }These things they who live a godly life, that is never to be repented of,
both have done and always will do.

\sectionnonum[Chapter LV]{55}{Examples of such love.}
To bring forward some examples from among the heathen: Many kings and princes, in times
of pestilence, when they had been instructed by an oracle, have given themselves up to death, in
order that by their own blood they might deliver their fellow-citizens [from destruction]. Many
have gone forth from their own cities, that so sedition might be brought to an end within them. We
know many among ourselves who have given themselves up to bonds, in order that they might
ransom others. Many, too, have surrendered themselves to slavery, that with the price\footnote{Literally, ``and having received their prices, fed others.`` [Comp. Rom. xvi.  3, 4, and Phil. ii.  30.] }which they
received for themselves, they might provide food for others. Many women also, being strengthened
by the grace of God, have performed numerous manly exploits. The blessed Judith, when her city
was besieged, asked of the elders permission to go forth into the camp of the strangers; and, exposing
herself to danger, she went out for the love which she bare to her country and people then besieged;
and the Lord delivered Holofernes into the hands of a woman.\footnote{Judith viii.  30.  }Esther also, being perfect in faith,
exposed herself to no less danger, in order to deliver the twelve tribes of Israel from impending
destruction. For with fasting and humiliation she entreated the everlasting God, who seeth all things;
and He, perceiving the humility of her spirit, delivered the people for whose sake she had encountered
peril.\footnote{Esth. vii., viii.}.

\sectionnonum[Chapter LVI]{56}{Let us admonish and correct one another.}
Let us then also pray for those who have fallen into any sin, that meekness and humility may
be given to them, so that they may submit, not unto us, but to the will of God. For in this way they
shall secure a fruitful and perfect remembrance from us, with sympathy for them, both in our prayers
to God, and our mention of them to the saints.\footnote{Literally, ``there shall be to them a fruitful and perfect remembrance, with compassions both towards God and the saints.`` }Let us receive correction, beloved, on account of
which no one should feel displeased. Those exhortations by which we admonish one another are
both good [in themselves] and highly profitable, for they tend to unite\footnote{Or, ``they unite.`` }us to the will of God. For
thus saith the holy Word: ``The Lord hath severely chastened me, yet hath not given me over to
death.``\footnote{Ps. cxviii.  18.  }``For whom the Lord loveth He chasteneth, and scourgeth every son whom He receiveth.``\footnote{Prov. iii.  12; Heb. xii.  6.  }``The righteous,`` saith it, ``shall chasten me in mercy, and reprove me; but let not the oil of sinners
make fat my head.``\footnote{Ps. cxli.  5.  }And again he saith, ``Blessed is the man whom the Lord reproveth, and reject
not thou the warning of the Almighty. For He causes sorrow, and again restores [to gladness]; He
woundeth, and His hands make whole. He shall deliver thee in six troubles, yea, in the seventh no
evil shall touch thee. In famine He shall rescue thee from death, and in war He shall free thee from
the power\footnote{Literally, ``hand.`` }of the sword. From the scourge of the tongue will He hide thee, and thou shalt not fear
when evil cometh. Thou shalt laugh at the unrighteous and the wicked, and shalt not be afraid of
the beasts of the field. For the wild beasts shall be at peace with thee: then shalt thou know that thy
house shall be in peace, and the habitation of thy tabernacle shall not fail.\footnote{Literally, ``err`` or ``sin.`` }Thou shall know also
that thy seed shall be great, and thy children like the grass of the field. And thou shall come to the
grave like ripened corn which is reaped in its season, or like a heap of the threshing-floor which is
gathered together at the proper time.``\footnote{Job v.  17 -  26.  }Ye see, beloved, that protection is afforded to those that
are chastened of the Lord; for since God is good, He corrects us, that we may be admonished by
His holy chastisement.

\sectionnonum[Chapter LVII]{57}{Let the authors of sedition submit themselves.}
Ye therefore, who laid the foundation of this sedition, submit yourselves to the presbyters, and
receive correction so as to repent, bending the knees of your hearts. Learn to be subject, laying
aside the proud and arrogant self-confidence of your tongue. For it is better for you that ye should
occupy\footnote{Literally, ``to be found small and esteemed.`` }a humble but honourable place in the flock of Christ, than that, being highly exalted, ye
should be cast out from the hope of His people.\footnote{Literally, ``His hope.`` [It has been conjectured that ἔλπιδος should be ἔπαύλιδος, and the reading, ``out of the fold of his people.`` See Chevallier.] }For thus speaketh all-virtuous Wisdom:\footnote{Prov. i.  23 -  31. [Often cited by this name in primitive writers.] }``Behold,
I will bring forth to you the words of My Spirit, and I will teach you My speech. Since I called,
and ye did not hear; I held forth My words, and ye regarded not, but set at naught My counsels,
and yielded not at My reproofs; therefore I too will laugh at your destruction; yea, I will rejoice
when ruin cometh upon you, and when sudden confusion overtakes you, when overturning presents
itself like a tempest, or when tribulation and oppression fall upon you. For it shall come to pass,
that when ye call upon Me, I will not hear you; the wicked shall seek Me, and they shall not find
Me. For they hated wisdom, and did not choose the fear of the Lord; nor would they listen to My
counsels, but despised My reproofs. Wherefore they shall eat the fruits of their own way, and they
shall be filled with their own ungodliness.`` \footnote{Junius (Pat. Young), who examined the MS. before it was bound into its present form, stated that a whole leaf was here lost. The next letters that occur are ιπον, which have been supposed to indicate εἶπον or ἔλιπον. Doubtless some passages quoted by the ancients from the Epistle of Clement, and not now found in it, occurred in the portion which has thus been lost.  }

\sectionnonum[Chapter LVIII]{58}{Blessings sought for all that call upon God.}
May God, who seeth all things, and who is the Ruler of all spirits and the Lord of all flesh - who
chose our Lord Jesus Christ and us through Him to be a peculiar\footnote{Comp. Tit. ii.  14.}people - grant to every soul that
calleth upon His glorious and holy Name, faith, fear, peace, patience, long-suffering, self-control,
purity, and sobriety, to the well-pleasing of His Name, through our High Priest and Protector, Jesus
Christ, by whom be to Him glory, and majesty, and power, and honour, both now and for evermore.
Amen.

\sectionnonum[Chapter LIX]{59}{The Corinthians are exhorted speedily to send back word that peace has been
restored. The benediction.}
Send back speedily to us in peace and with joy these our messengers to you: Claudius Ephebus
and Valerius Bito, with Fortunatus: that they may the sooner announce to us the peace and harmony
we so earnestly desire and long for [among you], and that we may the more quickly rejoice over
the good order re-established among you. The grace of our Lord Jesus Christ be with you, and with
all everywhere that are the called of God through Him, by whom be to Him glory, honour, power,
majesty, and eternal dominion,\footnote{Literally, ``an eternal throne.`` }from everlasting to everlasting.\footnote{Literally, ``From the ages to the ages of ages.`` }Amen.\footnote{[Note St. Clement’s frequent doxologies.] [N.B. - The language of Clement concerning the Western progress of St. Paul (cap. v.) is our earliest postscript to his Scripture biography. It is sufficient to refer the reader to the great works of Conybeare and Howson, and of Mr. Lewin, on the Life and Epistles of St. Paul. See more especially the valuable note of Lewin (vol. ii. p.  294) which takes notice of the opinion of some learned men, that the great Apostle of the Gentiles preached the Gospel in Britain.  The whole subject of St. Paul’s relations with British Christians is treated by Williams, in his Antiquities of the Cymry, with learning and in an attractive manner. But the reader will find more ready to his hand, perhaps, the interesting note of Mr. Lewin, on Claudia and Pudens ( 2 Tim. iv.  21), in his Life and Epistles of St. Paul, vol. ii. p.  392. See also Paley’s Hor{\ae} Paulin{\ae}, p.  40.  London, 1820.]}




31 Introductory Note to the Epistle of Polycarp to the Philippians
[A.D. 65–100–155.] THE Epistle of Polycarp is usually made a sort of preface to those of Ignatius,
for reasons which will be obvious to the reader. Yet he was born later, and lived to a much later
period. They seem to have been friends from the days of their common pupilage under St. John;
and there is nothing improbable in the conjecture of Usher, that he was the “angel of the church in
Smyrna,” to whom the Master says, “Be thou faithful unto death, and I will give thee a crown of
life.” His pupil Irenæus gives us one of the very few portraits of an apostolic man which are to be
found in antiquity, in a few sentences which are a picture: “I could describe the very place in which
the blessed Polycarp sat and taught; his going out and coming in; the whole tenor of his life; his
personal appearance; how he would speak of the conversations he had held with John and with
others who had seen the Lord. How did he make mention of their words and of whatever he had
heard from them respecting the Lord.” Thus he unconsciously tantalizes our reverent curiosity.
Alas! that such conversations were not written for our learning. But there is a wise Providence in
what is withheld, as well as in the inestimable treasures we have received.
Irenæus will tell us more concerning him, his visit to Rome, his rebuke of Marcion, and incidental
anecdotes, all which are instructive. The expression which he applied to Marcion is found in this
Epistle. Other facts of interest are found in the Martyrdom, which follows in these pages. His death,
in extreme old age under the first of the Antonines, has been variously dated; but we may accept
the date we have given, as rendered probable by that of the Paschal question, which he so lovingly
settled with Anicetus, Bishop of Rome.
The Epistle to the Philippians is the more interesting as denoting the state of that beloved church,
the firstborn of European churches, and so greatly endeared to St. Paul. It abounds in practical
wisdom, and is rich in Scripture and Scriptural allusions. It reflects the spirit of St. John, alike in
its lamb-like and its aquiline features: he is as loving as the beloved disciple himself when he speaks
of Christ and his church, but “the son of thunder” is echoed in his rebukes of threatened corruptions
in faith and morals. Nothing can be more clear than his view of the doctrines of grace; but he writes
like the disciple of St. John, though in perfect harmony with St. Paul’s hymn-like eulogy of Christian
love.
The following is the original INTRODUCTORY NOTICE:—
THE authenticity of the following Epistle can on no fair grounds be questioned. It is abundantly
established by external testimony, and is also supported by the internal evidence. Irenæus says
(Adv. Hær., iii. 3): “There is extant an Epistle of Polycarp written to the Philippians, most
55
ANF01. The Apostolic Fathers with Justin Martyr and Irenaeus Philip Schaff
32
satisfactory, from which those that have a mind to do so may learn the character of his faith,” etc.
This passage is embodied by Eusebius in his Ecclesiastical History (iv. 14); and in another place
the same writer refers to the Epistle before us as an undoubted production of Polycarp (Hist. Eccl.,
iii. 36). Other ancient testimonies might easily be added, but are superfluous, inasmuch as there is
a general consent among scholars at the present day that we have in this letter an authentic production
of the renowned Bishop of Smyrna.
Of Polycarp’s life little is known, but that little is highly interesting. Irenæus was his disciple,
and tells us that “Polycarp was instructed by the apostles, and was brought into contact with many
who had seen Christ” (Adv. Hær., iii. 3; Euseb. Hist. Eccl., iv. 14). There is also a very graphic
account given of Polycarp by Irenæus in his Epistle to Florinus, to which the reader is referred. It
has been preserved by Eusebius (Hist. Eccl., v. 20).
The Epistle before us is not perfect in any of the Greek MSS. which contain it. But the chapters
wanting in Greek are contained in an ancient Latin version. While there is no ground for supposing,
as some have done, that the whole Epistle is spurious, there seems considerable force in the
arguments by which many others have sought to prove chap. xiii. to be an interpolation.
The date of the Epistle cannot be satisfactorily determined. It depends on the conclusion we
reach as to some points, very difficult and obscure, connected with that account of the martyrdom
of Polycarp which has come down to us. We shall not, however, probably be far wrong if we fix
it about the middle of the second century.




POLYCARP, and the presbyters
\footnote{Or, ``Polycarp, and those who with him are presbyters.''}
with him, to the Church of God sojourning at Philippi: Mercy
to you, and peace from God Almighty, and from the Lord Jesus Christ, our Saviour, be multiplied.
\section{Chapter I.-Praise of the Philippians.}
I have greatly rejoiced with you in our Lord Jesus Christ, because ye have followed the example
\footnote{Literally, ``ye have received the patterns of true love.''}of true love [as displayed by God], and have accompanied, as became you, those who were bound
in chains, the fitting ornaments of saints, and which are indeed the diadems of the true elect of God
and our Lord; and because the strong root of your faith, spoken of in days
\footnote{Phil. i. 5.}long gone by, endureth

even until now, and bringeth forth fruit to our Lord Jesus Christ, who for our sins suffered even
unto death, [but] ``whom God raised from the dead, having loosed the bands of the grave''.
\footnote{Acts ii. 24. Literally, ``having loosed the pains of Hades.''}``In
whom, though now ye see Him not, ye believe, and believing, rejoice with joy unspeakable and
full of glory'';
\footnote{1 Pet. i. 8.}into which joy many desire to enter, knowing that ``by grace ye are saved, not of
works'',
\footnote{Eph. ii. 8, 9.}but by the will of God through Jesus Christ.
\section{Chapter II.-An exhortation to virtue.}
``Wherefore, girding up your loins'',
\footnote{Comp. 1 Pet. i. 13; Eph. vi. 14.}``serve the Lord in fea''r
\footnote{Ps. ii. 11.}and truth, as those who have
forsaken the vain, empty talk and error of the multitude, and ``believed in Him who raised up our
Lord Jesus Christ from the dead, and gave Him glory'',
\footnote{1 Pet. i. 21.}and a throne at His right hand. To Him
all things
\footnote{Comp. 1 Pet. iii. 22; Phil. ii. 10.}in heaven and on earth are subject. Him every spirit serves. He comes as the Judge of
the living and the dead.
\footnote{Comp. Acts xvii. 31.}His blood will God require of those who do not believe in Him.
\footnote{Or, ``who do not obey him.''}But
He who raised Him up from the dead will raise
\footnote{Comp 1 Cor. vi. 14; 2 Cor. iv. 14; Rom. viii. 11.}up us also, if we do His will, and walk in His
commandments, and love what He loved, keeping ourselves from all unrighteousness, covetousness,
love of money, evil speaking, false witness; ``not rendering evil for evil, or railing for railing'',
\footnote{1 Pet. iii. 9.}or blow for blow, or cursing for cursing, but being mindful of what the Lord said in His teaching:
``Judge not, that ye be not judged;
\footnote{Matt. vii. 1.}forgive, and it shall be forgiven unto you;
\footnote{Matt. vi. 12, 14; Luke vi. 37.}be merciful, that
ye may obtain mercy;
\footnote{Luke vi. 36.}with what measure ye mete, it shall be measured to you again'';
\footnote{Matt. vii. 2; Luke vi. 38.}and once





















more, ``Blessed are the poor, and those that are persecuted for righteousness’ sake, for theirs is the
kingdom of God''.
\footnote{Matt. v. 3, 10; Luke vi. 20.}\section{Chapter III.-Expressions of personal unworthiness.}
These things, brethren, I write to you concerning righteousness, not because I take anything
upon myself, but because ye have invited me to do so. For neither I, nor any other such one, can
come up to the wisdom
\footnote{Comp. 2 Pet. iii. 15.}of the blessed and glorified Paul. He, when among you, accurately and
stedfastly taught the word of truth in the presence of those who were then alive. And when absent
from you, he wrote you a letter,
\footnote{The form is plural, but one Epistle is probably meant. [So, even in English, ``letters'' may be classically used for a single letter, as we say ``by these presents.'' But even we might speak of St. Paul as having written his Epistles to us; so the Epistles to Thessalonica and Corinth might more naturally still be referred to here].}which, if you carefully study, you will find to be the means of
building you up in that faith which has been given you, and which, being followed by hope, and
preceded by love towards God, and Christ, and our neighbour, ``is the mother of us all''.
\footnote{Comp. Gal. iv. 26.}For if
any one be inwardly possessed of these graces, he hath fulfilled the command of righteousness,
since he that hath love is far from all sin.
\section{Chapter IV.-Various exhortations.} 
``But the love of money is the root of all evils''.
\footnote{1 Tim. vi. 10.}Knowing, therefore, that ``as we brought
nothing into the world, so we can carry nothing out'',
\footnote{1 Tim. vi. 7.}let us arm ourselves with the armour of
righteousness;
\footnote{Comp. Eph. vi. 11.}and let us teach, first of all, ourselves to walk in the commandments of the Lord.
Next, [teach] your wives [to walk] in the faith given to them, and in love and purity tenderly loving
their own husbands in all truth, and loving all [others] equally in all chastity; and to train up their
children in the knowledge and fear of God. Teach the widows to be discreet as respects the faith
of the Lord, praying continually
\footnote{Comp. 1 Thess. v. 17.}for all, being far from all slandering, evil-speaking,













false-witnessing, love of money, and every kind of evil; knowing that they are the altar
\footnote{Some here read, ``altars.''}of God,
that He clearly perceives all things, and that nothing is hid from Him, neither reasonings, nor
reflections, nor any one of the secret things of the heart.
\section{Chapter V.-The duties of deacons, youths, and virgins.}
Knowing, then, that ``God is not mocked'',
\footnote{Gal. vi. 7.}we ought to walk worthy of His commandment
and glory. In like manner should the deacons be blameless before the face of His righteousness, as
being the servants of God and Christ,
\footnote{Some read, ``God in Christ.''}and not of men. They must not be slanderers,
double-tongued,
\footnote{Comp. 1 Tim. iii. 8.}or lovers of money, but temperate in all things, compassionate, industrious,
walking according to the truth of the Lord, who was the servant
\footnote{Comp. Matt. xx. 28.}of all. If we please Him in this
present world, we shall receive also the future world, according as He has promised to us that He
will raise us again from the dead, and that if we live
\footnote{Πολιτευσώμεθα, referring to the whole conduct; comp. Phil. i. 27.}worthily of Him, ``we shall also reign together
with Him'',
\footnote{2 Tim. ii. 12.}provided only we believe. In like manner, let the young men also be blameless in all
things, being especially careful to preserve purity, and keeping themselves in, as with a bridle, from
every kind of evil. For it is well that they should be cut off from
\footnote{Some read, ἀνακύπτεσθαι, ``to emerge from.'' [So Chevallier, but not Wake nor Jacobson. See the note of latter, ad loc.]}the lusts that are in the world,
since ``every lust warreth against the spirit'';
\footnote{1 Pet. ii. 11.}and ``neither fornicators, nor effeminate, nor abusers
of themselves with mankind, shall inherit the kingdom of God'',
\footnote{1 Cor. vi. 9, 10.}nor those who do things
inconsistent and unbecoming. Wherefore, it is needful to abstain from all these things, being subject
to the presbyters and deacons, as unto God and Christ. The virgins also must walk in a blameless
and pure conscience.
\section{Chapter VI.-The duties of presbyters and others.}















And let the presbyters be compassionate and merciful to all, bringing back those that wander,
visiting all the sick, and not neglecting the widow, the orphan, or the poor, but always ``providing
for that which is becoming in the sight of God and man'';
\footnote{Rom. xii. 17; 2 Cor. viii. 31.}abstaining from all wrath, respect of
persons, and unjust judgment; keeping far off from all covetousness, not quickly crediting [an evil
report] against any one, not severe in judgment, as knowing that we are all under a debt of sin. If
then we entreat the Lord to forgive us, we ought also ourselves to forgive;
\footnote{Matt. vi. 12–14.}for we are before the
eyes of our Lord and God, and ``we must all appear at the judgment-seat of Christ, and must every
one give an account of himself''.
\footnote{Rom. xiv. 10–12; 2 Cor. v. 10.}Let us then serve Him in fear, and with all reverence, even as
He Himself has commanded us, and as the apostles who preached the Gospel unto us, and the
prophets who proclaimed beforehand the coming of the Lord [have alike taught us]. Let us be
zealous in the pursuit of that which is good, keeping ourselves from causes of offence, from false
brethren, and from those who in hypocrisy bear the name of the Lord, and draw away vain men
into error.
\section{Chapter VII.-Avoid the Docetæ, and persevere in fasting and prayer.}
``For whosoever does not confess that Jesus Christ has come in the flesh, is antichrist'';
\footnote{1 John iv. 3.}and
whosoever does not confess the testimony of the cross,
\footnote{Literally, ``the martyrdom of the cross,'' which some render, ``His suffering on the cross.''}is of the devil; and whosoever perverts
the oracles of the Lord to his own lusts, and says that there is neither a resurrection nor a judgment,
he is the first-born of Satan.
\footnote{[The original, perhaps, of Eusebius (Hist. iv. cap. 14). It became a common-place expression in the Church.]}Wherefore, forsaking the vanity of many, and their false doctrines,
let us return to the word which has been handed down to us from
\footnote{Comp. Jude 3.}the beginning; ``watching unto
prayer'',
\footnote{1 Pet. iv. 7.}and persevering in fasting; beseeching in our supplications the all-seeing God ``not to
lead us into temptation'',
\footnote{Matt. vi. 13; Matt. xxvi. 41.}as the Lord has said: ``The spirit truly is willing, but the flesh is weak''.
\footnote{Matt. xxvi. 41; Mark xiv. 38.}













\section{Chapter VIII.-Persevere in hope and patience.}
Let us then continually persevere in our hope, and the earnest of our righteousness, which is
Jesus Christ, ``who bore our sins in His own body on the tree'',
\footnote{1 Pet. ii. 24.}``who did no sin, neither was guile
found in His mouth'',
\footnote{1 Pet. ii. 22.}but endured all things for us, that we might live in Him.
\footnote{Comp. 1 John iv. 9.}Let us then be
imitators of His patience; and if we suffer
\footnote{Comp. Acts v. 41; 1 Pet. iv. 16.}for His name’s sake, let us glorify Him.
\footnote{Some read, ``we glorify Him.''}For He has
set us this example
\footnote{Comp. 1 Pet. ii. 21.}in Himself, and we have believed that such is the case.
\section{Chapter IX.-Patience inculcated.}
I exhort you all, therefore, to yield obedience to the word of righteousness, and to exercise all
patience, such as ye have seen [set] before your eyes, not only in the case of the blessed Ignatius,
and Zosimus, and Rufus, but also in others among yourselves, and in Paul himself, and the rest of
the apostles. [This do] in the assurance that all these have not run
\footnote{Comp. Phil. ii. 16; Gal. ii. 2.}in vain, but in faith and
righteousness, and that they are [now] in their due place in the presence of the Lord, with whom
also they suffered. For they loved not this present world, but Him who died for us, and for our sakes
was raised again by God from the dead.
\section{Chapter X.-Exhortation to the practice of virtue.}
\footnote{This and the two following chapters are preserved only in a Latin version. [See Jacobson, ad loc.]}Stand fast, therefore, in these things, and follow the example of the Lord, being firm and
unchangeable in the faith, loving the brotherhood,
\footnote{Comp. 1 Pet. ii. 17.}and being attached to one another, joined
together in the truth, exhibiting the meekness of the Lord in your intercourse with one another, and
despising no one. When you can do good, defer it not, because ``alms delivers from death''.
\footnote{Tobit iv. 10, Tobit xii. 9.}Be













all of you subject one to another
\footnote{Comp. 1 Pet. v. 5.}``having your conduct blameless among the Gentiles'',
\footnote{1 Pet. ii. 12.}that ye
may both receive praise for your good works, and the Lord may not be blasphemed through you.
But woe to him by whom the name of the Lord is blasphemed!
\footnote{Isa. lii. 5.}Teach, therefore, sobriety to all,
and manifest it also in your own conduct.
\section{Chapter XI.-Expression of grief on account of Valens.}
I am greatly grieved for Valens, who was once a presbyter among you, because he so little
understands the place that was given him [in the Church]. I exhort you, therefore, that ye abstain
from covetousness,
\footnote{Some think that incontinence on the part of the Valens and his wife is referred to. [For many reasons I am glad the translators have preferred the reading πλεονεξίας. The next word, chaste, sufficiently rebukes the example of Valens. For once I venture not to coincide with Jacobson’s comment.]}and that ye be chaste and truthful. ``Abstain from every form of evil''.
\footnote{1 Thess. v. 22.}For
if a man cannot govern himself in such matters, how shall he enjoin them on others? If a man does
not keep himself from covetousness,
\footnote{Some think that incontinence on the part of the Valens and his wife is referred to. [For many reasons I am glad the translators have preferred the reading πλεονεξίας. The next word, chaste, sufficiently rebukes the example of Valens. For once I venture not to coincide with Jacobson’s comment.]}he shall be defiled by idolatry, and shall be judged as one
of the heathen. But who of us are ignorant of the judgment of the Lord? ``Do we not know that the
saints shall judge the world''?
\footnote{1 Cor. vi. 2.}as Paul teaches. But I have neither seen nor heard of any such thing
among you, in the midst of whom the blessed Paul laboured, and who are commended
\footnote{Some read, ``named;'' comp. Phil. i. 5.}in the
beginning of his Epistle. For he boasts of you in all those Churches which alone then knew the
Lord; but we [of Smyrna] had not yet known Him. I am deeply grieved, therefore, brethren, for
him (Valens) and his wife; to whom may the Lord grant true repentance! And be ye then moderate
in regard to this matter, and ``do not count such as enemies'',
\footnote{2 Thess. iii. 15.}but call them back as suffering and
straying members, that ye may save your whole body. For by so acting ye shall edify yourselves.
\footnote{Comp. 1 Cor. xii. 26.}













\section{Chapter XII.-Exhortation to various graces.}
For I trust that ye are well versed in the Sacred Scriptures, and that nothing is hid from you;
but to me this privilege is not yet granted.
\footnote{This passage is very obscure. Some render it as follows: ``But at present it is not granted unto me to practise that which is written, Be ye angry,'' etc.}It is declared then in these Scriptures, ``Be ye angry,
and sin not'',
\footnote{Ps. iv. 5.}and, ``Let not the sun go down upon your wrath''.
\footnote{Eph. iv. 26.}Happy is he who remembers
\footnote{Some read, ``believes.''}this, which I believe to be the case with you. But may the God and Father of our Lord Jesus Christ,
and Jesus Christ Himself, who is the Son of God, and our everlasting High Priest, build you up in
faith and truth, and in all meekness, gentleness, patience, long-suffering, forbearance, and purity;
and may He bestow on you a lot and portion among His saints, and on us with you, and on all that
are under heaven, who shall believe in our Lord Jesus Christ, and in His Father, who ``raised Him
from the dead''.
\footnote{Gal. i. 1.}Pray for all the saints. Pray also for kings,
\footnote{Comp. 1 Tim. ii. 2.}and potentates, and princes, and for
those that persecute and hate you,
\footnote{Matt. v. 44.}and for the enemies of the cross, that your fruit may be manifest
to all, and that ye may be perfect in Him.
\section{Chapter XIII.-Concerning the transmission of epistles.}
Both you and Ignatius
\footnote{Comp. Ep. of Ignatius to Polycarp, chap. viii.}wrote to me, that if any one went [from this] into Syria, he should
carry your letter
\footnote{Or, ``letters.''}with him; which request I will attend to if I find a fitting opportunity, either
personally, or through some other acting for me, that your desire may be fulfilled. The Epistles of
Ignatius written by him
\footnote{Reference is here made to the two letters of Ignatius, one to Polycarp himself, and the other to the church at Smyrna.}to us, and all the rest [of his Epistles] which we have by us, we have sent
to you, as you requested. They are subjoined to this Epistle, and by them ye may be greatly profited;
for they treat of faith and patience, and all things that tend to edification in our Lord. Any
\footnote{Henceforth, to the end, we have only the Latin version.}more
certain information you may have obtained respecting both Ignatius himself, and those that were
(<>)415
with him, have the goodness to make known
(<>)416 to us.
\section{Chapter XIV.-Conclusion.}
These things I have written to you by Crescens, whom up to the present
(<>)417 time I have
recommended unto you, and do now recommend. For he has acted blamelessly among us, and I
believe also among you. Moreover, ye will hold his sister in esteem when she comes to you. Be ye
safe in the Lord Jesus Christ. Grace be with you all.
(<>)418 Amen.

37 Introductory Note to the Epistle Concerning the Martyrdom of Polycarp
INTERNAL evidence goes far to establish the credit which Eusebius lends to this specimen of the
martyrologies, certainly not the earliest if we accept that of Ignatius as genuine. As an encyclical
of one of “the seven churches” to another of the same Seven, and as bearing witness to their
aggregation with others into the unity of “the Holy and Catholic Church,” it is a very interesting
witness, not only to an article of the creed, but to the original meaning and acceptation of the same.
More than this, it is evidence of the strength of Christ perfected in human weakness; and thus it
affords us an assurance of grace equal to our day in every time of need. When I see in it, however,
an example of what a noble army of martyrs, women and children included, suffered in those days
“for the testimony of Jesus,” and in order to hand down the knowledge of the Gospel to these
boastful ages of our own, I confess myself edified by what I read, chiefly because I am humbled
and abashed in comparing what a Christian used to be, with what a Christian is, in our times, even
at his best estate.
That this Epistle has been interpolated can hardly be doubted, when we compare it with the
unvarnished specimen, in Eusebius. As for the “fragrant smell” that came from the fire, many kinds
of wood emit the like in burning; and, apart from Oriental warmth of colouring, there seems nothing
incredible in the narrative if we except “the dove” (chap. xvi.), which, however, is probably a
415 The Latin version reads “are,” which has been corrected as above.
416 Polycarp was aware of the death of Ignatius (chap. ix.), but was as yet apparently ignorant of the circumstances attending
it. [Who can fail to be touched by these affectionate yet entirely calm expressions as to his martyred friend and brother? Martyrdom
was the habitual end of Christ’s soldiers, and Polycarp expected his own; hence his restrained and temperate words of interest.]
417 Some read, “in this present Epistle.”
418 Others read, “and in favour with all yours.”
64
ANF01. The Apostolic Fathers with Justin Martyr and Irenaeus Philip Schaff
corrupt reading,419 as suggested by our translators. The blade was thrust into the martyr’s left side;
and this, opening the heart, caused the outpouring of a flood, and not a mere trickling. But, though
Greek thus amended is a plausible conjecture, there seems to have been nothing of the kind in the
copy quoted by Eusebius. On the other hand, note the truly catholic and scriptural testimony: “We
love the martyrs, but the Son of God we worship: it is impossible for us to worship any other.”
Bishop Jacobson assigns more than fifty pages to this martyrology, with a Latin version and
abundant notes. To these I must refer the student, who may wish to see this attractive history in all
the light of critical scholarship and, often, of admirable comment.
The following is the original INTRODUCTORY NOTICE:—
THE following letter purports to have been written by the Church at Smyrna to the Church at
Philomelium, and through that Church to the whole Christian world, in order to give a succinct
account of the circumstances attending the martyrdom of Polycarp. It is the earliest of all the
Martyria, and has generally been accounted both the most interesting and authentic. Not a few,
38
however, deem it interpolated in several passages, and some refer it to a much later date than the
middle of the second century, to which it has been commonly ascribed. We cannot tell how much
it may owe to the writers (chap. xxii.) who successively transcribed it. Great part of it has been
engrossed by Eusebius in his Ecclesiastical History (iv. 15); and it is instructive to observe, that
some of the most startling miraculous phenomena recorded in the text as it now stands, have no
place in the narrative as given by that early historian of the Church. Much discussion has arisen
respecting several particulars contained in this Martyrium; but into these disputes we do not enter,
having it for our aim simply to present the reader with as faithful a translation as possible of this
very interesting monument of Christian antiquity.


\chapter{The Encyclical Epistle of the Church at Smyrna Concerning the Martyrdom
of the Holy Polycarp}
THE Church of God which sojourns at Smyrna, to the Church of God sojourning in
Philomelium,
\footnote{Some read, ``Philadelphia,'' but on inferior authority. Philomelium was a city of Phrygia.}and to all the congregations
\footnote{The word in the original is ποροικίαις, from which the English ``parishes'' is derived.}of the Holy and Catholic Church in every place:
Mercy, peace, and love from God the Father, and our Lord Jesus Christ, be multiplied.
\section{Chapter I.- Subject of which we write.\protect\footnote{See an ingenious conjecture in Bishop Wordsworth’s Hippolytus and the Church of Rome, p. 318, C.}}




We have written to you, brethren, as to what relates to the martyrs, and especially to the blessed
Polycarp, who put an end to the persecution, having, as it were, set a seal upon it by his martyrdom.
For almost all the events that happened previously [to this one], took place that the Lord might
show us from above a martyrdom becoming the Gospel. For he waited to be delivered up, even as
the Lord had done, that we also might become his followers, while we look not merely at what
concerns ourselves but have regard also to our neighbours. For it is the part of a true and
well-founded love, not only to wish one’s self to be saved, but also all the brethren.
\section{Chapter II.- The wonderful constancy of the martyrs.}
All the martyrdoms, then, were blessed and noble which took place according to the will of
God. For it becomes us who profess
\footnote{Literally, ``who are more pious.''}greater piety than others, to ascribe the authority over all
things to God. And truly,
\footnote{The account now returns to the illustration of the statement made in the first sentence.}who can fail to admire their nobleness of mind, and their patience, with
that love towards their Lord which they displayed?- who, when they were so torn with scourges,
that the frame of their bodies, even to the very inward veins and arteries, was laid open, still patiently
endured, while even those that stood by pitied and bewailed them. But they reached such a pitch
of magnanimity, that not one of them let a sigh or a groan escape them; thus proving to us all that
those holy martyrs of Christ, at the very time when they suffered such torments, were absent from
the body, or rather, that the Lord then stood by them, and communed with them. And, looking to
the grace of Christ, they despised all the torments of this world, redeeming themselves from eternal
punishment by [the suffering of] a single hour. For this reason the fire of their savage executioners
appeared cool to them. For they kept before their view escape from that fire which is eternal and
never shall be quenched, and looked forward with the eyes of their heart to those good things which
are laid up for such as endure; things ``which ear hath not heard, nor eye seen, neither have entered
into the heart of man'',
\footnote{1 Cor. ii. 9.}but were revealed by the Lord to them, inasmuch as they were no longer
men, but had already become angels. And, in like manner, those who were condemned to the wild
beasts endured dreadful tortures, being stretched out upon beds full of spikes, and subjected to
various other kinds of torments, in order that, if it were possible, the tyrant might, by their lingering
tortures, lead them to a denial [of Christ].
\section{Chapter III.- The constancy of Germanicus. The death of Polycarp is demanded.}







For the devil did indeed invent many things against them; but thanks be to God, he could not
prevail over all. For the most noble Germanicus strengthened the timidity of others by his own
patience, and fought heroically
\footnote{Or, ``illustriously.''}with the wild beasts. For, when the proconsul sought to persuade
him, and urged him
\footnote{Or, ``said to him.''}to take pity upon his age, he attracted the wild beast towards himself, and
provoked it, being desirous to escape all the more quickly from an unrighteous and impious world.
But upon this the whole multitude, marvelling at the nobility of mind displayed by the devout and
godly race of Christians,
\footnote{Literally, ``the nobleness of the God-loving and God-fearing race of Christians.''}cried out, ``Away with the Atheists; let Polycarp be sought out''!
\section{Chapter IV.- Quintus the apostate.}
Now one named Quintus, a Phrygian, who was but lately come from Phrygia, when he saw the
wild beasts, became afraid. This was the man who forced himself and some others to come forward
voluntarily [for trial]. Him the proconsul, after many entreaties, persuaded to swear and to offer
sacrifice. Wherefore, brethren, we do not commend those who give themselves up [to suffering],
seeing the Gospel does not teach so to do.
\footnote{Comp. Matt. x. 23.}\section{Chapter V.- The departure and vision of Polycarp.}
But the most admirable Polycarp, when he first heard [that he was sought for], was in no measure
disturbed, but resolved to continue in the city. However, in deference to the wish of many, he was
persuaded to leave it. He departed, therefore, to a country house not far distant from the city. There
he stayed with a few [friends], engaged in nothing else night and day than praying for all men, and
for the Churches throughout the world, according to his usual custom. And while he was praying,
a vision presented itself to him three days before he was taken; and, behold, the pillow under his
head seemed to him on fire. Upon this, turning to those that were with him, he said to them
prophetically, ``I must be burnt alive''.
\section{Chapter VI.- Polycarp is betrayed by a servant.}








And when those who sought for him were at hand, he departed to another dwelling, whither
his pursuers immediately came after him. And when they found him not, they seized upon two
youths [that were there], one of whom, being subjected to torture, confessed. It was thus impossible
that he should continue hid, since those that betrayed him were of his own household. The Irenarch
\footnote{It was the duty of the Irenarch to apprehend all seditious troublers of the public peace.}then (whose office is the same as that of the Cleronomus
\footnote{Some think that those magistrates bore this name that were elected by lot.}), by name Herod, hastened to bring him
into the stadium. [This all happened] that he might fulfil his special lot, being made a partaker of
Christ, and that they who betrayed him might undergo the punishment of Judas himself.
\section{Chapter VII.- Polycarp is found by his pursuers.}
His pursuers then, along with horsemen, and taking the youth with them, went forth at
supper-time on the day of the preparation
\footnote{That is, on Friday.}with their usual weapons, as if going out against a
robber.
\footnote{Comp. Matt. xxvi. 55.}And being come about evening [to the place where he was], they found him lying down
in the upper room of
\footnote{Or, ``in.''}a certain little house, from which he might have escaped into another place;
but he refused, saying, ``The will of God
\footnote{Some read ``the Lord''}be done''.
\footnote{Comp. Matt. vi. 10; Acts xxi. 14.}So when he heard that they were come, he
went down and spake with them. And as those that were present marvelled at his age and constancy,
some of them said. ``Was so much effort
\footnote{Or, ``diligence.''}made to capture such a venerable man''?
\footnote{Jacobson reads, ``and [marvelling] that they had used so great diligence to capture,'' etc.}Immediately
then, in that very hour, he ordered that something to eat and drink should be set before them, as
much indeed as they cared for, while he besought them to allow him an hour to pray without
disturbance. And on their giving him leave, he stood and prayed, being full of the grace of God, so
that he could not cease
\footnote{Or, ``be silent.''}for two full hours, to the astonishment of them that heard him, insomuch
that many began to repent that they had come forth against so godly and venerable an old man.
\section{Chapter VIII.- Polycarp is brought into the city.}













Now, as soon as he had ceased praying, having made mention of all that had at any time come
in contact with him, both small and great, illustrious and obscure, as well as the whole Catholic
Church throughout the world, the time of his departure having arrived, they set him upon an ass,
and conducted him into the city, the day being that of the great Sabbath. And the Irenarch Herod,
accompanied by his father Nicetes (both riding in a chariot
\footnote{Jacobson deems these words an interpolation.}), met him, and taking him up into the
chariot, they seated themselves beside him, and endeavoured to persuade him, saying, ``What harm
is there in saying, Lord Cæsar,
\footnote{Or, ``Cæsar is Lord,'' all the MSS. having κύριος instead of κύριε, as usually printed.}and in sacrificing, with the other ceremonies observed on such
occasions, and so make sure of safety?'' But he at first gave them no answer; and when they continued
to urge him, he said, ``I shall not do as you advise me.'' So they, having no hope of persuading him,
began to speak bitter
\footnote{Or, ``terrible.''}words unto him, and cast him with violence out of the chariot,
\footnote{Or, ``cast him down'' simply, the following words being, as above, an interpolation.}insomuch
that, in getting down from the carriage, he dislocated his leg
\footnote{Or, ``sprained his ankle.''}[by the fall]. But without being
disturbed,
\footnote{Or, ``not turning back.''}and as if suffering nothing, he went eagerly forward with all haste, and was conducted
to the stadium, where the tumult was so great, that there was no possibility of being heard.
\section{Chapter IX.- Polycarp refuses to revile Christ.}
Now, as Polycarp was entering into the stadium, there came to him a voice from heaven, saying,
``Be strong, and show thyself a man, O Polycarp!'' No one saw who it was that spoke to him; but
those of our brethren who were present heard the voice. And as he was brought forward, the tumult
became great when they heard that Polycarp was taken. And when he came near, the proconsul
asked him whether he was Polycarp. On his confessing that he was, [the proconsul] sought to
persuade him to deny [Christ], saying, ``Have respect to thy old age,'' and other similar things,
according to their custom, [such as], ``Swear by the fortune of Cæsar; repent, and say, Away with
the Atheists.'' But Polycarp, gazing with a stern countenance on all the multitude of the wicked
heathen then in the stadium, and waving his hand towards them, while with groans he looked up
to heaven, said, ``Away with the Atheists''.
\footnote{Referring the words to the heathen, and not to the Christians, as was desired.}Then, the proconsul urging him, and saying, ``Swear,
and I will set thee at liberty, reproach Christ;'' Polycarp declared, ``Eighty and six years have I
served Him, and He never did me any injury: how then can I blaspheme my King and my Saviour''?









\section{Chapter X.- Polycarp confesses himself a Christian.}
And when the proconsul yet again pressed him, and said, ``Swear by the fortune of Cæsar,'' he
answered, ``Since thou art vainly urgent that, as thou sayest, I should swear by the fortune of Cæsar,
and pretendest not to know who and what I am, hear me declare with boldness, I am a Christian.
And if you wish to learn what the doctrines
\footnote{Or, ``an account of Christianity.''}of Christianity are, appoint me a day, and thou shalt
hear them.'' The proconsul replied, ``Persuade the people.'' But Polycarp said, ``To thee I have
thought it right to offer an account [of my faith]; for we are taught to give all due honour (which
entails no injury upon ourselves) to the powers and authorities which are ordained of God.
\footnote{Comp. Rom. xiii. 1–7; Tit. iii. 1.}But
as for these, I do not deem them worthy of receiving any account from me''.
\footnote{Or, ``of my making any defence to them.''}\section{Chapter XI.- No threats have any effect on Polycarp.}
The proconsul then said to him, ``I have wild beasts at hand; to these will I cast thee, except
thou repent.'' But he answered, ``Call them then, for we are not accustomed to repent of what is
good in order to adopt that which is evil;
\footnote{Literally, ``repentance from things better to things worse is a change impossible to us.''}and it is well for me to be changed from what is evil to
what is righteous''.
\footnote{That is, to leave this world for a better.}But again the proconsul said to him, ``I will cause thee to be consumed by
fire, seeing thou despisest the wild beasts, if thou wilt not repent.'' But Polycarp said, ``Thou
threatenest me with fire which burneth for an hour, and after a little is extinguished, but art ignorant
of the fire of the coming judgment and of eternal punishment, reserved for the ungodly. But why
tarriest thou? Bring forth what thou wilt''.
\section{Chapter XII.- Polycarp is sentenced to be burned.}
While he spoke these and many other like things, he was filled with confidence and joy, and
his countenance was full of grace, so that not merely did it not fall as if troubled by the things said
to him, but, on the contrary, the proconsul was astonished, and sent his herald to proclaim in the
midst of the stadium thrice, ``Polycarp has confessed that he is a Christian.'' This proclamation
having been made by the herald, the whole multitude both of the heathen and Jews, who dwelt at








Smyrna, cried out with uncontrollable fury, and in a loud voice, ``This is the teacher of Asia,
\footnote{Some read, ``ungodliness,'' but the above seems preferable.}the
father of the Christians, and the overthrower of our gods, he who has been teaching many not to
sacrifice, or to worship the gods.'' Speaking thus, they cried out, and besought Philip the Asiarch
\footnote{The Asiarchs were those who superintended all arrangements connected with the games in the several provinces.}to let loose a lion upon Polycarp. But Philip answered that it was not lawful for him to do so, seeing
the shows
\footnote{Literally, ``the baiting of dogs.''}of wild beasts were already finished. Then it seemed good to them to cry out with one
consent, that Polycarp should be burnt alive. For thus it behooved the vision which was revealed
to him in regard to his pillow to be fulfilled, when, seeing it on fire as he was praying, he turned
about and said prophetically to the faithful that were with him, ``I must be burnt alive''.
\section{Chapter XIII.- The funeral pile is erected.}
This, then, was carried into effect with greater speed than it was spoken, the multitudes
immediately gathering together wood and fagots out of the shops and baths; the Jews especially,
according to custom, eagerly assisting them in it. And when the funeral pile was ready, Polycarp,
laying aside all his garments, and loosing his girdle, sought also to take off his sandals,- a thing
he was not accustomed to do, inasmuch as every one of the faithful was always eager who should
first touch his skin. For, on account of his holy life,
\footnote{Literally, ``good behaviour.''}he was, even before his martyrdom, adorned
\footnote{Some think this implies that Polycarp’s skin was believed to possess a miraculous efficacy.}with every kind of good. Immediately then they surrounded him with those substances which had
been prepared for the funeral pile. But when they were about also to fix him with nails, he said,
``Leave me as I am; for He that giveth me strength to endure the fire, will also enable me, without
your securing me by nails, to remain without moving in the pile''.
\section{Chapter XIV.- The prayer of Polycarp.}
They did not nail him then, but simply bound him. And he, placing his hands behind him, and
being bound like a distinguished ram [taken] out of a great flock for sacrifice, and prepared to be
an acceptable burnt-offering unto God, looked up to heaven, and said, ``O Lord God Almighty, the
Father of thy beloved and blessed Son Jesus Christ, by whom we have received the knowledge of
Thee, the God of angels and powers, and of every creature, and of the whole race of the righteous
who live before thee, I give Thee thanks that Thou hast counted me, worthy of this day and this








hour, that I should have a part in the number of Thy martyrs, in the cup
\footnote{Comp. Matt. xx. 22, Matt. xxvi. 39; Mark x. 38.}of thy Christ, to the
resurrection of eternal life, both of soul and body, through the incorruption [imparted] by the Holy
Ghost. Among whom may I be accepted this day before Thee as a fat
\footnote{Literally, ``in a fat,'' etc., [or, ``in a rich''].}and acceptable sacrifice,
according as Thou, the ever-truthful
\footnote{Literally, ``the not false and true God.''}God, hast foreordained, hast revealed beforehand to me, and
now hast fulfilled. Wherefore also I praise Thee for all things, I bless Thee, I glorify Thee, along
with the everlasting and heavenly Jesus Christ, Thy beloved Son, with whom, to Thee, and the
Holy Ghost, be glory both now and to all coming ages. Amen''.
\footnote{Eusebius (Hist. Eccl., iv. 15) has preserved a great portion of this Martyrium, but in a text considerably differing from that we have followed. Here, instead of ``and,'' he has ``in the Holy Ghost.''}\section{Chapter XV.- Polycarp is not injured by the fire.}
When he had pronounced this amen, and so finished his prayer, those who were appointed for
the purpose kindled the fire. And as the flame blazed forth in great fury,
\footnote{Literally, ``a great flame shining forth.''}we, to whom it was
given to witness it, beheld a great miracle, and have been preserved that we might report to others
what then took place. For the fire, shaping itself into the form of an arch, like the sail of a ship
when filled with the wind, encompassed as by a circle the body of the martyr. And he appeared
within not like flesh which is burnt, but as bread that is baked, or as gold and silver glowing in a
furnace. Moreover, we perceived such a sweet odour [coming from the pile], as if frankincense or
some such precious spices had been smoking
\footnote{Literally, ``breathing.''}there.
\section{Chapter XVI.- Polycarp is pierced by a dagger.}
At length, when those wicked men perceived that his body could not be consumed by the fire,
they commanded an executioner to go near and pierce him through with a dagger. And on his doing
this, there came forth a dove,
\footnote{Eusebius omits all mention of the dove, and many have thought the text to be here corrupt. It has been proposed to read ἐπ’ ἀριστερᾷ, ``on the left hand side,'' instead of περιστερά, ``a dove.''}and a great quantity of blood, so that the fire was extinguished; and
all the people wondered that there should be such a difference between the unbelievers and the
elect, of whom this most admirable Polycarp was one, having in our own times been an apostolic








and prophetic teacher, and bishop of the Catholic Church which is in Smyrna. For every word that
went out of his mouth either has been or shall yet be accomplished.
\section{Chapter XVII.- The Christians are refused Polycarp’s body.}
But when the adversary of the race of the righteous, the envious, malicious, and wicked one,
perceived the impressive
\footnote{Literally, ``greatness.''}nature of his martyrdom, and [considered] the blameless life he had led
from the beginning, and how he was now crowned with the wreath of immortality, having beyond
dispute received his reward, he did his utmost that not the least memorial of him should be taken
away by us, although many desired to do this, and to become possessors
\footnote{The Greek, literally translated, is, ``and to have fellowship with his holy flesh.''}of his holy flesh. For
this end he suggested it to Nicetes, the father of Herod and brother of Alce, to go and entreat the
governor not to give up his body to be buried, ``lest,'' said he, ``forsaking Him that was crucified,
they begin to worship this one.'' This he said at the suggestion and urgent persuasion of the Jews,
who also watched us, as we sought to take him out of the fire, being ignorant of this, that it is neither
possible for us ever to forsake Christ, who suffered for the salvation of such as shall be saved
throughout the whole world (the blameless one for sinners
\footnote{This clause is omitted by Eusebius: it was probably interpolated by some transcriber, who had in his mind 1 Pet. iii. 18.}), nor to worship any other. For Him
indeed, as being the Son of God, we adore; but the martyrs, as disciples and followers of the Lord,
we worthily love on account of their extraordinary
\footnote{Literally, ``unsurpassable.''}affection towards their own King and Master,
of whom may we also be made companions
\footnote{Literally, ``fellow-partakers.''}and fellow-disciples!
\section{Chapter XVIII.- The body of Polycarp is burned.}
The centurion then, seeing the strife excited by the Jews, placed the body
\footnote{Or, ``him.''}in the midst of the
fire, and consumed it. Accordingly, we afterwards took up his bones, as being more precious than
the most exquisite jewels, and more purified
\footnote{Or, ``more tried.''}than gold, and deposited them in a fitting place,
whither, being gathered together, as opportunity is allowed us, with joy and rejoicing, the Lord
shall grant us to celebrate the anniversary
\footnote{Literally, ``the birth-day.''}of his martyrdom, both in memory of those who have











already finished their course,
\footnote{Literally, ``been athletes.''}and for the exercising and preparation of those yet to walk in their
steps.
\section{Chapter XIX.- Praise of the martyr Polycarp.}
This, then, is the account of the blessed Polycarp, who, being the twelfth that was martyred in
Smyrna (reckoning those also of Philadelphia), yet occupies a place of his own
\footnote{Literally, ``is alone remembered.''}in the memory
of all men, insomuch that he is everywhere spoken of by the heathen themselves. He was not merely
an illustrious teacher, but also a pre-eminent martyr, whose martyrdom all desire to imitate, as
having been altogether consistent with the Gospel of Christ. For, having through patience overcome
the unjust governor, and thus acquired the crown of immortality, he now, with the apostles and all
the righteous [in heaven], rejoicingly glorifies God, even the Father, and blesses our Lord Jesus
Christ, the Saviour of our souls, the Governor of our bodies, and the Shepherd of the Catholic
Church throughout the world.
\footnote{Several additions are here made. One MS. has, ``and the all-holy and life-giving Spirit;'' while the old Latin version reads, ``and the Holy Spirit, by whom we know all things.''}\section{Chapter XX.- This epistle is to be transmitted to the brethren.}
Since, then, ye requested that we would at large make you acquainted with what really took
place, we have for the present sent you this summary account through our brother Marcus. When,
therefore, ye have yourselves read this Epistle,
\footnote{Literally, ``having learned these things.''}be pleased to send it to the brethren at a greater
distance, that they also may glorify the Lord, who makes such choice of His own servants. To Him
who is able to bring us all by His grace and goodness
\footnote{Literally, ``gift.''}into his everlasting kingdom, through His
only-begotten Son Jesus Christ, to Him be glory, and honour, and power, and majesty, for ever.
Amen. Salute all the saints. They that are with us salute you, and Evarestus, who wrote this Epistle,
with all his house.
\section{Chapter XXI.- The date of the martyrdom.}
Now, the blessed Polycarp suffered martyrdom on the second day of the month Xanthicus just
begun,
\footnote{Great obscurity hangs over the chronology here indicated. According to Usher, the Smyrnæans began the month Xanthicus 
on the 25th of March. But the seventh day before the Kalends of May is the 25th of April. 
Some, therefore, read ᾽Απριλλίων instead of Μαίων. The great Sabbath is that before the passover. 
The ``eighth hour'' may correspond either to our 8 A.M. or 2 P.M.}
the seventh day before the Kalends of May, on the great Sabbath, at the eighth hour. He was taken by Herod, Philip the Trallian being high priest, \footnote{Called before (chap. xii.) Asiarch.}Statius Quadratus being proconsul,
but Jesus Christ being King for ever, to whom be glory, honour, majesty, and an everlasting throne,
from generation to generation. Amen.
\section{Chapter XXII.- Salutation.}
We wish you, brethren, all happiness, while you walk according to the doctrine of the Gospel
of Jesus Christ; with whom be glory to God the Father and the Holy Spirit, for the salvation of His
holy elect, after whose example
\footnote{Literally, ``according as.''}the blessed Polycarp suffered, following in whose steps may we
too be found in the kingdom of Jesus Christ!
These things
\footnote{What follows is, of course, no part of the original Epistle.}Caius transcribed from the copy of Irenæus (who was a disciple of Polycarp),
having himself been intimate with Irenæus. And I Socrates transcribed them at Corinth from the
copy of Caius. Grace be with you all.
And I again, Pionius, wrote them from the previously written copy, having carefully searched
into them, and the blessed Polycarp having manifested them to me through a revelation, even as
I shall show in what follows. I have collected these things, when they had almost faded away
through the lapse of time, that the Lord Jesus Christ may also gather me along with His elect into
His heavenly kingdom, to whom, with the Father and the Holy Spirit, be glory for ever and ever.
Amen.


\partnonum{1}{Introductory Note to the Epistle of Mathetes to Diognetus [A.D. 130.]}
THE anonymous author of this Epistle gives himself the title (Mathetes) “a disciple 
\footnote{\textgreek{apost\'{o}lwn gen\'omeno ma}\texttheta\textgreek{hths}. Cap. xi.} of the Apostles,” and I venture to adopt it as his name. It is about all we know of him, and it serves
a useful end. I place his letter here, as a sequel to the Clementine Epistle, for several reasons, which
I think scholars will approve: (1) It is full of the Pauline spirit, and exhales the same pure and
primitive fragrance which is characteristic of Clement. (2) No theory as to its date very much
conflicts with that which I adopt, and it is sustained by good authorities. (3) But, as a specimen of
the persuasives against Gentilism which early Christians employed in their intercourse with friends
who adhered to heathenism, it admirably illustrates the temper prescribed by St. Paul (2 Tim. ii.
24), and not less the peculiar social relations of converts to the Gospel with the more amiable and
candid of their personal friends at this early period.
Mathetes was possibly a catechumen of St. Paul or of one of the apostle’s associates. I assume
that his correspondent was the tutor of M. Aurelius. Placed just here, it fills a lacuna in the series,
and takes the place of the pseudo (second) Epistle of Clement, which is now relegated to its proper
place with the works falsely ascribed to St. Clement.
Altogether, the Epistle is a gem of purest ray; and, while suggesting some difficulties as to
interpretation and exposition, it is practically clear as to argument and intent. Mathetes is, perhaps,
the first of the apologists.



\partnonum{2}{The following is the original INTRODUCTORY NOTICE of the learned editors and translators}
THE following interesting and eloquent Epistle is anonymous, and we have no clue whatever
as to its author. For a considerable period after its publication in 1592, it was generally ascribed to
Justin Martyr. In recent times Otto has inserted it among the works of that writer, but Semisch and
others contend that it cannot possibly be his. In dealing with this question, we depend entirely upon
the internal evidence, no statement as to the authorship of the Epistle having descended to us from
antiquity. And it can scarcely be denied that the whole tone of the Epistle, as well as special passages
which it contains, points to some other writer than Justin. Accordingly, critics are now for the most
part agreed that it is not his, and that it must be ascribed to one who lived at a still earlier date in
the history of the Church. Several internal arguments have been brought forward in favour of this
opinion. Supposing chap. xi. to be genuine, it has been supported by the fact that the writer there
styles himself “a disciple of the apostles.” But there is great suspicion that the two concluding

chapters are spurious; and even though admitted to be genuine, the expression quoted evidently
admits of a different explanation from that which implies the writer’s personal acquaintance with
the apostles: it might, indeed, be adopted by one even at the present day. More weight is to be
attached to those passages in which the writer speaks of Christianity as still being a new thing in
the world. Expressions to this effect occur in several places (chap. i., ii., ix.), and seem to imply
that the author lived very little, if at all, after the apostolic age. There is certainly nothing in the
Epistle which is inconsistent with this opinion; and we may therefore believe, that in this beautiful
composition we possess a genuine production of some apostolic man who lived not later than the
beginning of the second century.
The names of Clement of Rome and of Apollos have both been suggested as those of the probable
author. Such opinions, however, are pure fancies, which it is perhaps impossible to refute, but which
rest on nothing more than conjecture. Nor can a single word be said as to the person named
Diognetus, to whom the letter is addressed. We must be content to leave both points in hopeless
obscurity, and simply accept the Epistle as written by an earnest and intelligent Christian to a sincere
inquirer among the Gentiles, towards the close of the apostolic age.
It is much to be regretted that the text is often so very doubtful. Only three MSS. of the Epistle,
all probably exhibiting the same original text, are known to exist; and in not a few passages the
readings are, in consequence, very defective and obscure. But notwithstanding this drawback, and
the difficulty of representing the full force and elegance of the original, this Epistle, as now presented
to the English reader, can hardly fail to excite both his deepest interest and admiration.
[N.B.—Interesting speculations concerning this precious work may be seen in Bunsen’s
Hippolytus and his Age, vol. i. p. 188. The learned do not seem convinced by this author, but I have
adopted his suggestion as to Diognetus the tutor of M. Aurelius.]


\partnonum{1}{The Epistle of Mathetes to Diognetus}


\sectionnonum[Chapter I]{1}{Occasion of the epistle}
SINCE I see thee, most excellent Diognetus, exceedingly desirous to learn the mode of worshipping
God prevalent among the Christians, and inquiring very carefully and earnestly concerning them,
what God they trust in, and what form of religion they observe,\footnote{Literally, “trusting in what God, etc., they look down.” the world,} so as all to look down upon the
world itself, and despise death, while they neither esteem those to be gods that are reckoned such
by the Greeks, nor hold to the superstition of the Jews; and what is the affection which they cherish
among themselves; and why, in fine, this new kind or practice [of piety] has only now entered into\footnote{Or, “life,”} and not long ago; I cordially welcome this thy desire, and I implore God, who enables us both to speak and to hear, to grant to me so to speak, that, above all, I may hear you have been edified,\footnote{Some read, “that you by hearing may be edified.”} and to you so to hear, that I who speak may have no cause of regret for having done so.

\sectionnonum[Chapter II]{2}{The vanity of idols}
Come, then, after you have freed\footnote{Or, “purified.”} yourself from all prejudices possessing your mind, and laid
aside what you have been accustomed to, as something apt to deceive\footnote{Literally, “which is deceiving.”} you, and being made, as
if from the beginning, a new man, inasmuch as, according to your own confession, you are to be
the hearer of a new [system of] doctrine; come and contemplate, not with your eyes only, but with
your understanding, the substance and the form\footnote{Literally, “of what substance, or of what form.”} of those whom ye declare and deem to be gods.
Is not one of them a stone similar to that on which we tread? Is\footnote{Some make this and the following clauses affirmative instead of interrogative. } not a second brass, in no way
superior to those vessels which are constructed for our ordinary use? Is not a third wood, and that
already rotten? Is not a fourth silver, which needs a man to watch it, lest it be stolen? Is not a fifth
iron, consumed by rust? Is not a sixth earthenware, in no degree more valuable than that which is
formed for the humblest purposes? Are not all these of corruptible matter? Are they not fabricated
by means of iron and fire? Did not the sculptor fashion one of them, the brazier a second, the
silversmith a third, and the potter a fourth? Was not every one of them, before they were formed
by the arts of these [workmen] into the shape of these [gods], each in its\footnote{The text is here corrupt. Several attempts at emendation have been made, but without any marked success.} own way subject to
change? Would not those things which are now vessels, formed of the same materials, become like
to such, if they met with the same artificers? Might not these, which are now worshipped by you,
again be made by men vessels similar to others? Are they not all deaf? Are they not blind? Are
they not without life? Are they not destitute of feeling? Are they not incapable of motion? Are they
not all liable to rot? Are they not all corruptible? These things ye call gods; these ye serve; these
ye worship; and ye become altogether like to them. For this reason ye hate the Christians, because
they do not deem these to be gods. But do not ye yourselves, who now think and suppose [such to
be gods], much more cast contempt upon them than they [the Christians do]? Do ye not much more
mock and insult them, when ye worship those that are made of stone and earthenware, without
appointing any persons to guard them; but those made of silver and gold ye shut up by night, and
appoint watchers to look after them by day, lest they be stolen? And by those gifts which ye mean
to present to them, do ye not, if they are possessed of sense, rather punish [than honour] them? But
if, on the other hand, they are destitute of sense, ye convict them of this fact, while ye worship them
with blood and the smoke of sacrifices. Let any one of you suffer such indignities!\footnote{Some read, “Who of you would tolerate these things?” etc. } Let any one
of you endure to have such things done to himself! But not a single human being will, unless
compelled to it, endure such treatment, since he is endowed with sense and reason. A stone,
however, readily bears it, seeing it is insensible. Certainly you do not show [by your\footnote{The text is here uncertain, and the sense obscure. The meaning seems to be, that by sprinkling their gods with blood, etc., they tended to prove that these were not possessed of sense. } conduct]
that he [your God] is possessed of sense. And as to the fact that Christians are not accustomed to
serve such gods, I might easily find many other things to say; but if even what has been said does
not seem to any one sufficient, I deem it idle to say anything further.


\sectionnonum[Chapter III]{3}{Superstitions of the Jews}
And next, I imagine that you are most desirous of hearing something on this point, that the
Christians do not observe the same forms of divine worship as do the Jews. The Jews, then, if they
abstain from the kind of service above described, and deem it proper to worship one God as being
Lord of all, [are right]; but if they offer Him worship in the way which we have described, they
greatly err. For while the Gentiles, by offering such things to those that are destitute of sense and
hearing, furnish an example of madness; they, on the other hand by thinking to offer these things
to God as if He needed them, might justly reckon it rather an act of folly than of divine worship.
For He that made heaven and earth, and all that is therein, and gives to us all the things of which
we stand in need, certainly requires none of those things which He Himself bestows on such as
think of furnishing them to Him. But those who imagine that, by means of blood, and the smoke
of sacrifices and burnt-offerings, they offer sacrifices [acceptable] to Him, and that by such honours
they show Him respect, —these, by\footnote{The text here is very doubtful. We have followed that adopted by most critics.} supposing that they can give anything to Him who stands in
need of nothing, appear to me in no respect to differ from those who studiously confer the same
honour on things destitute of sense, and which therefore are unable to enjoy such honours.

\sectionnonum[Chapter IV]{4}{The other observances of the Jews}
But as to their scrupulosity concerning meats, and their superstition as respects the Sabbaths,
and their boasting about circumcision, and their fancies about fasting and the new moons, which
are utterly ridiculous and unworthy of notice,—I do not\footnote{Otto, resting on MS. authority, omits the negative, but the sense seems to require its insertion. } think that you require to learn anything
from me. For, to accept some of those things which have been formed by God for the use of men
as properly formed, and to reject others as useless and redundant,—how can this be lawful? And
to speak falsely of God, as if He forbade us to do what is good on the Sabbath-days,—how is not
this impious? And to glory in the circumcision\footnote{Literally, “lessening.”} of the flesh as a proof of election, and as if, on
account of it, they were specially beloved by God,—how is it not a subject of ridicule? And as to
their observing months and days,\footnote{Comp. Gal. iv.  10. } as if waiting upon\footnote{This seems to refer to the practice of Jews in fixing the beginning of the day, and consequently of the Sabbath, from the rising of the stars. They used to say, that when three stars of moderate magnitude appeared, it was night; when two, it was twilight; and when only one, that day had not yet departed. It thus came to pass (according to their night-day (\textgreek{nuk}\texttheta\textgreek{\'{h} meron}) reckoning), that whosoever engaged in work on the evening of Friday, the beginning of the Sabbath, after three stars of moderate size were visible, was held to have sinned, and had to present a trespass-offering; and so on, according to the fanciful rule described. } the stars and the moon, and their
distributing,\footnote{Otto supplies the lacuna which here occurs in the MSS. so as to read \textgreek{katadiaire\~{i} n}.} according to their own tendencies, the appointments of God, and the vicissitudes of
the seasons, some for festivities,\footnote{The great festivals of the Jews are here referred to on the one hand, and the day of atonement on the other.} and others for mourning,—who would deem this a part of divine
worship, and not much rather a manifestation of folly? I suppose, then, you are sufficiently convinced
that the Christians properly abstain from the vanity and error common [to both Jews and Gentiles],
and from the busy-body spirit and vain boasting of the Jews; but you must not hope to learn the
mystery of their peculiar mode of worshipping God from any mortal.



\sectionnonum[Chapter V]{5}{The manners of the Christians}
For the Christians are distinguished from other men neither by country, nor language, nor the
customs which they observe. For they neither inhabit cities of their own, nor employ a peculiar
form of speech, nor lead a life which is marked out by any singularity. The course of conduct which
they follow has not been devised by any speculation or deliberation of inquisitive men; nor do they,
like some, proclaim themselves the advocates of any merely human doctrines. But, inhabiting Greek
as well as barbarian cities, according as the lot of each of them has determined, and following the
customs of the natives in respect to clothing, food, and the rest of their ordinary conduct, they
display to us their wonderful and confessedly striking\footnote{Literally, “paradoxical.”} method of life. They dwell in their own
countries, but simply as sojourners. As citizens, they share in all things with others, and yet endure
all things as if foreigners. Every foreign land is to them as their native country, and every land of
their birth as a land of strangers. They marry, as do all [others]; they beget children; but they do
not destroy their offspring.\footnote{Literally, “cast away fetuses.”} They have a common table, but not a common bed.\footnote{Otto omits “bed,” which is an emendation, and gives the second “common” the sense of unclean. } They are in
the flesh, but they do not live after the flesh.\footnote{Comp.  2 Cor. x.  3.} They pass their days on earth, but they are citizens
of heaven.\footnote{Comp. Phil. iii.  20. } They obey the prescribed laws, and at the same time surpass the laws by their lives.
They love all men, and are persecuted by all. They are unknown and condemned; they are put to
death, and restored to life.\footnote{Comp.  2 Cor. vi.  9. } They are poor, yet make many rich;\footnote{Comp.  2 Cor. vi.  10. } they are in lack of all things,
and yet abound in all; they are dishonoured, and yet in their very dishonour are glorified. They are
evil spoken of, and yet are justified; they are reviled, and bless;\footnote{Comp.  2 Cor. iv.  12. } they are insulted, and repay the
insult with honour; they do good, yet are punished as evil-doers. When punished, they rejoice as
if quickened into life; they are assailed by the Jews as foreigners, and are persecuted by the Greeks;
yet those who hate them are unable to assign any reason for their hatred.


\sectionnonum[Chapter VI]{6}{The relation of Christians to the world}
To sum up all in one word— what the soul is in the body, that are Christians in the world. The
soul is dispersed through all the members of the body, and Christians are scattered through all the
cities of the world. The soul dwells in the body, yet is not of the body; and Christians dwell in the
world, yet are not of the world.\footnote{John xvii.  11, 14, 16. } The invisible soul is guarded by the visible body, and Christians
are known indeed to be in the world, but their godliness remains invisible. The flesh hates the soul,
and wars against it,\footnote{Comp.  1 Pet. ii.  11.} though itself suffering no injury, because it is prevented from enjoying
pleasures; the world also hates the Christians, though in nowise injured, because they abjure
pleasures. The soul loves the flesh that hates it, and [loves also] the members; Christians likewise
love those that hate them. The soul is imprisoned in the body, yet preserves\footnote{Literally, “keeps together.”} that very body; and
Christians are confined in the world as in a prison, and yet they are the preservers\footnote{Literally, “keeps together.”} of the world.
The immortal soul dwells in a mortal tabernacle; and Christians dwell as sojourners in corruptible
[bodies], looking for an incorruptible dwelling\footnote{Literally, “incorruption.”} in the heavens. The soul, when but ill-provided
with food and drink, becomes better; in like manner, the Christians, though subjected day by day
to punishment, increase the more in number.\footnote{Or, “though punished, increase in number daily.”} God has assigned them this illustrious position,
which it were unlawful for them to forsake.



\sectionnonum[Chapter VII]{7}{The manifestation of Christ}
For, as I said, this was no mere earthly invention which was delivered to them, nor is it a mere
human system of opinion, which they judge it right to preserve so carefully, nor has a dispensation
of mere human mysteries been committed to them, but truly God Himself, who is almighty, the
Creator of all things, and invisible, has sent from heaven, and placed among men, [Him who is]
the truth, and the holy and incomprehensible Word, and has firmly established Him in their hearts.
He did not, as one might have imagined, send to men any servant, or angel, or ruler, or any one of
those who bear sway over earthly things, or one of those to whom the government of things in the
heavens has been entrusted, but the very Creator and Fashioner of all things—by whom He made
the heavens—by whom he enclosed the sea within its proper bounds—whose ordinances\footnote{Literally, “mysteries.”} all the
stars\footnote{Literally, “elements.”} faithfully observe—from whom the sun\footnote{The word “sun,” though omitted in the MSS., should manifestly be inserted. } has received the measure of his daily course to
be observed\footnote{Literally, “has received to observe.”}— whom the moon obeys, being commanded to shine in the night, and whom the
stars also obey, following the moon in her course; by whom all things have been arranged, and
placed within their proper limits, and to whom all are subject—the heavens and the things that are
therein, the earth and the things that are therein, the sea and the things that are therein—fire, air,
and the abyss—the things which are in the heights, the things which are in the depths, and the things
which lie between. This [messenger] He sent to them. Was it then, as one\footnote{Literally, “one of men.”} might conceive, for
the purpose of exercising tyranny, or of inspiring fear and terror? By no means, but under the
influence of clemency and meekness. As a king sends his son, who is also a king, so sent He Him;
as God\footnote{“God” here refers to the person sent. } He sent Him; as to men He sent Him; as a Saviour He sent Him, and as seeking to persuade,
not to compel us; for violence has no place in the character of God. As calling us He sent Him, not
as vengefully pursuing us; as loving us He sent Him, not as judging us. For He will yet send Him
to judge us, and who shall endure His appearing?\footnote{[Comp. Mal. iii.  2. The Old Testament is frequently in mind, if not expressly quoted by Mathetes.] A considerable gap here occurs in the MSS. }… Do you not see them exposed to wild beasts,
that they may be persuaded to deny the Lord, and yet not overcome? Do you not see that the more
of them are punished, the greater becomes the number of the rest? This does not seem to be the
work of man: this is the power of God; these are the evidences of His manifestation.



\sectionnonum[Chapter VIII]{8}{The miserable state of men before the coming of the Word}
For, who of men at all understood before His coming what God is? Do you accept of the vain
and silly doctrines of those who are deemed trustworthy philosophers? of whom some said that fire
was God, calling that God to which they themselves were by and by to come; and some water; and
others some other of the elements formed by God. But if any one of these theories be worthy of
approbation, every one of the rest of created things might also be declared to be God. But such
declarations are simply the startling and erroneous utterances of deceivers;\footnote{Literally, “these things are the marvels and error.”} and no man has either
seen Him, or made Him known,\footnote{Or, “known Him.”} but He has revealed Himself. And He has manifested Himself
through faith, to which alone it is given to behold God. For God, the Lord and Fashioner of all
things, who made all things, and assigned them their several positions, proved Himself not merely
a friend of mankind, but also long-suffering [in His dealings with them]. Yea, He was always of
such a character, and still is, and will ever be, kind and good, and free from wrath, and true, and
the only one who is [absolutely] good;\footnote{Comp. Matt. xix.  17. } and He formed in His mind a great and unspeakable
conception, which He communicated to His Son alone. As long, then, as He held and preserved
His own wise counsel in concealment,\footnote{Literally, “in a mystery.”} He appeared to neglect us, and to have no care over us.
But after He revealed and laid open, through His beloved Son, the things which had been prepared
from the beginning, He conferred every blessing\footnote{Literally, “all things.”} all at once upon us, so that we should both share
in His benefits, and see and be active\footnote{The sense is here very obscure. We have followed the text of Otto, who fills up the lacuna in the MS. as above. Others have, “to see, and to handle Him.”}[in His service]. Who of us would ever have expected these
things? He was aware, then, of all things in His own mind, along with His Son, according to the
relation\footnote{Literally, “economically.”} subsisting between them.



\sectionnonum[Chapter IX]{9}{Why the Son was sent so late}
As long then as the former time\footnote{Otto refers for a like contrast between these two times to Rom. iii.  21– 26, Rom. v.  20 and Gal. iv.  4. [Comp. Acts xvii.  30.]} endured, He permitted us to be borne along by unruly impulses,
being drawn away by the desire of pleasure and various lusts. This was not that He at all delighted
in our sins, but that He simply endured them; nor that He approved the time of working iniquity
which then was, but that He sought to form a mind conscious of righteousness,\footnote{The reading and sense are doubtful. } so that being
convinced in that time of our unworthiness of attaining life through our own works, it should now,
through the kindness of God, be vouchsafed to us; and having made it manifest that in ourselves
we were unable to enter into the kingdom of God, we might through the power of God be made
able. But when our wickedness had reached its height, and it had been clearly shown that its
reward,\footnote{Both the text and rendering are here somewhat doubtful, but the sense will in any case be much the same. } punishment and death, was impending over us; and when the time had come which God
had before appointed for manifesting His own kindness and power, how\footnote{Many variations here occur in the way in which the lacuna of the MSS. is to be supplied. They do not, however, greatly affect the meaning. } the one love of God,
through exceeding regard for men, did not regard us with hatred, nor thrust us away, nor remember
our iniquity against us, but showed great long-suffering, and bore with us,\footnote{In the MS. “saying” is here inserted, as if the words had been regarded as a quotation from Isa. liii.  11.} He Himself took on
Him the burden of our iniquities, He gave His own Son as a ransom for us, the holy One for
transgressors, the blameless One for the wicked, the righteous One for the unrighteous, the
incorruptible One for the corruptible, the immortal One for them that are mortal. For what other
thing was capable of covering our sins than His righteousness? By what other one was it possible
that we, the wicked and ungodly, could be justified, than by the only Son of God? O sweet exchange!
O unsearchable operation! O benefits surpassing all expectation! that the wickedness of many
should be hid in a single righteous One, and that the righteousness of One should justify many
transgressors!\footnote{[See Bossuet, who quotes it as from Justin Martyr (Tom. iii. p.  171). Sermon on Circumcision.]} Having therefore convinced us in the former time\footnote{That is, before Christ appeared. } that our nature was unable to
attain to life, and having now revealed the Saviour who is able to save even those things which it
was [formerly] impossible to save, by both these facts He desired to lead us to trust in His kindness,
to esteem Him our Nourisher, Father, Teacher, Counsellor, Healer, our Wisdom, Light, Honour,
Glory, Power, and Life, so that we should not be anxious\footnote{Comp. Matt. vi.  25, etc. [Mathetes, in a single sentence, expounds a most practical text with comprehensive views.]} concerning clothing and food.
\sectionnonum[Chapter X]{10}{The blessings that will flow from faith}
If you also desire [to possess] this faith, you likewise shall receive first of all the knowledge of
the Father.\footnote{Thus Otto supplies the lacuna; others conjecture somewhat different supplements. } For God has loved mankind, on whose account He made the world, to whom He
rendered subject all the things that are in it,\footnote{So Böhl. Sylburgius and Otto read, “in the earth.”} to whom He gave reason and understanding, to whom
alone He imparted the privilege of looking upwards to Himself, whom He formed after His own
image, to whom He sent His only-begotten Son, to whom He has promised a kingdom in heaven,
and will give it to those who have loved Him. And when you have attained this knowledge, with
what joy do you think you will be filled? Or, how will you love Him who has first so loved you?
And if you love Him, you will be an imitator of His kindness. And do not wonder that a man may
become an imitator of God. He can, if he is willing. For it is not by ruling over his neighbours, or
by seeking to hold the supremacy over those that are weaker, or by being rich, and showing violence
towards those that are inferior, that happiness is found; nor can any one by these things become an
imitator of God. But these things do not at all constitute His majesty. On the contrary he who takes
upon himself the burden of his neighbour; he who, in whatsoever respect he may be superior, is
ready to benefit another who is deficient; he who, whatsoever things he has received from God, by
distributing these to the needy, becomes a god to those who receive [his benefits]: he is an imitator
of God. Then thou shalt see, while still on earth, that God in the heavens rules over [the universe];
then thou shall begin to speak the mysteries of God; then shalt thou both love and admire those
that suffer punishment because they will not deny God; then shall thou condemn the deceit and
error of the world when thou shall know what it is to live truly in heaven, when thou shalt despise
that which is here esteemed to be death, when thou shalt fear what is truly death, which is reserved
for those who shall be condemned to the eternal fire, which shall afflict those even to the end that
are committed to it. Then shalt thou admire those who for righteousness’ sake endure the fire that
is but for a moment, and shalt count them happy when thou shalt know [the nature of] that fire.






\sectionnonum[Chapter XI]{11}{These things are worthy to be known and believed}
I do not speak of things strange to me, nor do I aim at anything inconsistent with right reason;\footnote{Some render, “nor do I rashly seek to persuade others.”} but having been a disciple of the Apostles, I am become a teacher of the Gentiles. I minister the
things delivered to me to those that are disciples worthy of the truth. For who that is rightly taught
and begotten by the loving\footnote{Some propose to read, “and becoming a friend to the Word.”} Word, would not seek to learn accurately the things which have been
clearly shown by the Word to His disciples, to whom the Word being manifested has revealed them,
speaking plainly [to them], not understood indeed by the unbelieving, but conversing with the
disciples, who, being esteemed faithful by Him, acquired a knowledge of the mysteries of the
Father? For which\footnote{It has been proposed to connect this with the preceding sentence, and read, “have known the mysteries of the Father, viz., for what purpose He sent the Word.”} reason He sent the Word, that He might be manifested to the world; and He,
being despised by the people [of the Jews], was, when preached by the Apostles, believed on by
the Gentiles.\footnote{[Comp.  1 Tim. iii.  16.]} This is He who was from the beginning, who appeared as if new, and was found
old, and yet who is ever born afresh in the hearts of the saints. This is He who, being from everlasting,
is to-day called\footnote{Or, “esteemed.”} the Son; through whom the Church is enriched, and grace, widely spread, increases
in the saints, furnishing understanding, revealing mysteries, announcing times, rejoicing over the
faithful, giving\footnote{Or, “given.”} to those that seek, by whom the limits of faith are not broken through, nor the
boundaries set by the fathers passed over. Then the fear of the law is chanted, and the grace of the
prophets is known, and the faith of the gospels is established, and the tradition of the Apostles is
preserved, and the grace of the Church exults; which grace if you grieve not, you shall know those
things which the Word teaches, by whom He wills, and when He pleases. For whatever things we
are moved to utter by the will of the Word commanding us, we communicate to you with pains,
and from a love of the things that have been revealed to us.






\sectionnonum[Chapter XII]{12}{The importance of knowledge to true spiritual life}
When you have read and carefully listened to these things, you shall know what God bestows
on such as rightly love Him, being made [as ye are] a paradise of delight, presenting\footnote{Literally, “bringing forth.”} in yourselves
a tree bearing all kinds of produce and flourishing well, being adorned with various fruits. For in
this place\footnote{That is, in Paradise. } the tree of knowledge and the tree of life have been planted; but it is not the tree of
knowledge that destroys— it is disobedience that proves destructive. Nor truly are those words
without significance which are written, how God from the beginning planted the tree of life in the
midst of paradise, revealing through knowledge the way to life,\footnote{Literally “revealing life.”} and when those who were first
formed did not use this [knowledge] properly, they were, through the fraud of the Serpent, stripped
naked.\footnote{Or, “deprived of it.”} For neither can life exist without knowledge, nor is knowledge secure without life.
Wherefore both were planted close together. The Apostle, perceiving the force [of this conjunction],
and blaming that knowledge which, without true doctrine, is admitted to influence life,\footnote{Literally, “knowledge without the truth of a command exercised to life.” See 1 Cor. viii.  1. } declares,
“Knowledge puffeth up, but love edifieth.” For he who thinks he knows anything without true
knowledge, and such as is witnessed to by life, knows nothing, but is deceived by the Serpent, as
not\footnote{The MS. is here defective. Some read, “on account of the love of life.”} loving life. But he who combines knowledge with fear, and seeks after life, plants in hope,
looking for fruit. Let your heart be your wisdom; and let your life be true knowledge\footnote{Or, “true word,” or “reason.”} inwardly
received. Bearing this tree and displaying its fruit, thou shalt always gather\footnote{Or, “reap.”} in those things which
are desired by God, which the Serpent cannot reach, and to which deception does not approach;
nor is Eve then corrupted,\footnote{The meaning seems to be, that if the tree of true knowledge and life be planted within you, you shall continue free from blemishes and sins. } but is trusted as a virgin; and salvation is manifested, and the Apostles
are filled with understanding, and the Passover\footnote{[This looks like a reference to the Apocalypse, Rev. v.  9., Rev. xix.  7., Rev. xx.  5.]} of the Lord advances, and the choirs\footnote{Here Bishop Wordsworth would read \textgreek{kl\~{h} roi}, cites 1 Pet. v.  3, and refers to Suicer (Lexicon) in voce \textgreek{kl\~{h} roz}.]} are gathered
together, and are arranged in proper order, and the Word rejoices in teaching the saints,—by whom
the Father is glorified: to whom be glory for ever. Amen.\footnote{[Note the Clement-like doxology.]}


%backmatter: appendix, bibliography, index, postface

%%%front cover, spline, and back cover

\end{document}
